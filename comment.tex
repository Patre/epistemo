% !TEX encoding = UTF-8 Unicode
% !TEX root = rapport.tex

\chapter{Intégration / Solutions : Quelles sont les initiatives mises en place pour contrer le phénomène de débilisation des individus par l'informatique}\label{initiatives_actuelles}

\section{Initiatives des pouvoirs publics}

\subsection{Mesures des pouvoirs publics français}
\subsubsection{Mesures en faveur de la prise en main de l'outil informatique}
\cite{b2i_c2i}
\cite{b2i}
\cite{isn}

\subsubsection{opérations "portables" visant à l'accès du plus grand nombre à l'informatique}
\cite{portables35}
\cite{portables60}
\cite{portables40}

    


\subsection{Mesures des pouvoirs publics internationaux}
Forum mondial sur l’éducation \cite{educ_forum}
Les TIC au service de l’éducation \cite{tics}
Un regard sur la trajectoire de l’informatique éducative au Brésil \cite{peixoto2006regard}
Willem J. PELGRUM, Arian T. SCHIPPER, "Indicators of computer integration in education" \cite{pelgrum1993indicators}



\section{Initiatives des acteurs privés}



\section{Quelles peuvent être les solutions adaptées pour endiguer le phénomène ?}\label{solutions}

Interactions entre professeur et élèves
Révision des programmes
Utilisation des nouvelles technologies au service des interactions
Changements des règles de constitution des classes : pourquoi par tranches d'âge ?
Personnalisation des parcours en fonction des envies des élèves.


