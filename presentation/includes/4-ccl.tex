% !TEX root = ../apprentissage.tex
\section{Conclusion}

\begin{frame}{Conclusion}
  \begin{itemize}
    \item de tout temps l'évolution technique à eu des détracteur
    \item l'éducation connait un décalage avec les besoins sociétal (nivean TIC)
    \item ce décalage empêche l'arrivé des jeunes sur le marché de l'emploi
    \item ce décalage favorise l'échec scolaire
    \item ainsi que des drames liée à la mauvaise utilisation des NTIC
    \item depuis le début de 20ème siècle de nombreuse initiative sont née
    \item 
  \end{itemize}
\end{frame}

\begin{frame}{Concrètement}
\begin{itemize}
  \item D'abord confronter l'élève à la problématique avant de lui enseigner les concepts abstraits
  \item Réaliser des projets plutôt que résoudre des problèmes
  \item 
\end{itemize}
\end{frame}

\begin{frame}{Ouverture}
  \vfill
    \begin{block}{Neurosciences}
    \begin{itemize}
    \item Faire un résumé en fin de cours
    \item Création de carte mentale
    \end{itemize}
  \end{block}
  \vfill
  \begin{center}
    \includegraphics[width=.6\textwidth]{../resources/illustrations/mindmap.jpg}
  \end{center}
    \vfill
  \footlineextra{\cite{neurosup}}
\end{frame}

\begin{frame}
\Huge
\begin{center}
	Merci pour votre attention !
	
	Avez-vous des questions ?
	\includegraphicsabsolute{../resources/illustrations/seymour_skinner_2}{4cm}{.3cm}{6cm}
\end{center}
\end{frame}

