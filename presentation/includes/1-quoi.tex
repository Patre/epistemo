% !TEX root = ../apprentissage.tex

\section{Informatique, éducation et société}

%--------------------------------------------------------------------------------
\begin{frame}{À propos de \ldots}
\begin{coolquote}
Elle ne peut produire dans les âmes, en effet, que l’oubli de ce qu’elles savent en leur faisant négliger la mémoire.
\end{coolquote}
\end{frame}
%--------------------------------------------------------------------------------
\begin{frame}{À propos de l'écriture}
\begin{coolquote}[Socrates / Platon, Phaedrus 370BC \footlineextra{\cite{plato-phaedrus}}]
Elle ne peut produire dans les âmes, en effet, que l’oubli de ce qu’elles savent en leur faisant négliger la mémoire [\ldots] Tu as trouvé le moyen, non point d’enrichir la mémoire, mais de conserver les souvenirs qu’elle a.
\end{coolquote}
\end{frame}
%--------------------------------------------------------------------------------
% PLATO VS. WRITING
%--------------------------------------------------------------------------------
\begin{frame}{Pendant l'antiquité}
\begin{columns}
	\begin{column}{0.40\linewidth}
		\centering
		\includegraphics[height=0.35\paperheight]{../resources/illustrations/plato} \\
	\end{column}
	\begin{column}{0.09\linewidth} \centering \huge \emph vs. \end{column}
	\begin{column}{0.40\linewidth}
		\centering
		\includegraphics[height=0.35\paperheight]{../resources/illustrations/histories-fragment} \\
	\end{column}
\end{columns}
\begin{columns}
	\begin{column}{0.49\linewidth}
		\begin{itemize}
			\item Traditions orales~: épopées\footlineextra{\cite{havelock-preface-plato}}
			\item Dialectique / Maïeutique / Rhétorique
		\end{itemize}
		$\implies$ \textbf{Mémorisation, assimilation, débat.}\footlineextra{\cite{plato-phaedrus}}
	\end{column}
	\begin{column}{0.49\linewidth}
		\begin{itemize}
			\item Invention de l'écriture ($\approx 3500BC$)
			\item La Bible~: parole de Dieu
		\end{itemize}
		$\implies$ \textbf{Mémoire externe, consultation, standardisation.}
	\end{column}
\end{columns}
\end{frame}
%--------------------------------------------------------------------------------
\begin{frame}

\centering
\Huge \emph{À quoi sert donc le par-c\oe{}ur?}
\vfill
\centering
\includegraphics[height=0.5\paperheight]{../resources/illustrations/bloom} \\
\vfill \hfill
\large Taxonomie de Bloom (1956)\footlineextra{\cite{tax-bloom}} \hfill \hfill
\end{frame}

%--------------------------------------------------------------------------------
% MONKS VERSUS PRINTING
%--------------------------------------------------------------------------------
\begin{frame}{À propos de \ldots}
\begin{coolquote}
Donnons des ailes à la vérité [\ldots] par une machine infatigable.
\end{coolquote}
\end{frame}
%--------------------------------------------------------------------------------
\begin{frame}{À propos de la presse}
\begin{coolquote}[Gutenberg, 1455]
Donnons des ailes à la vérité, qu'elle ne soit plus manuscrite à grands frais par des mains qui se fatiguent, mais qu'ils volent multipliés par une machine infatigable et qu'ils atteignent tous les hommes.
\end{coolquote}
\end{frame}
%--------------------------------------------------------------------------------
\begin{frame}{Pendant le moyen-âge}
\begin{columns}
	\begin{column}{0.40\linewidth}
		\centering
		\includegraphics[height=0.35\paperheight]{../resources/illustrations/lecture} \\
	\end{column}
	\begin{column}{0.09\linewidth} \centering \huge \emph vs. \end{column}
	\begin{column}{0.40\linewidth}
		\centering
		\includegraphics[height=0.35\paperheight]{../resources/illustrations/gutenberg} \\
	\end{column}
\end{columns}
\begin{columns}
	\begin{column}{0.49\linewidth}
		\begin{itemize}
			\item Cours magistraux
			\item Amanuensis (moines copistes)
		\end{itemize}
		$\implies$ \textbf{Prise de notes, vérité unique.}
	\end{column}
	\begin{column}{0.49\linewidth}
		\begin{itemize}
			\item Jikjii au Corée (1377)
			\item Bible de Gutenberg en Europe (1440)
		\end{itemize}
		$\implies$ \textbf{Copiage mécanique, diffusion de idées.}
	\end{column}
\footlineextra{\cite{walsham2003}}
\footlineextra{\cite{nodier1989}}
\footlineextra{\cite{liberation2009}} 
\end{columns}
\end{frame}
%--------------------------------------------------------------------------------
\begin{frame}
\centering
\huge \emph{À quoi sert donc la prise de notes?}
\vfill
\centering
\includegraphics[height=0.5\paperheight]{../resources/illustrations/cours_magistral} \\
\vfill \hfill
\large Dessin de Francis Haraux (2007).\hfill \hfill
\end{frame}

%--------------------------------------------------------------------------------
% LUNAR DISTANCE VERSUS HARRISON
%--------------------------------------------------------------------------------
\begin{frame}{À propos de\ldots}
\begin{coolquote}
Un artifice admirable qui, en réduisant à quelques jours un travail de plusieurs mois, double la durée de vie[\ldots].
\end{coolquote}
\end{frame}
%--------------------------------------------------------------------------------
\begin{frame}{À propos des logarithmes}
\begin{coolquote}[Laplace, 1614 \footlineextra{\cite{history-of-astronomy}}]
Un artifice admirable qui, en réduisant à quelques jours un travail de plusieurs mois, double la durée de vie de l'astronome, et lui épargne les erreurs et le dégoût : plaies inséparables des longs calculs.
\end{coolquote}
\end{frame}
%--------------------------------------------------------------------------------
\begin{frame}{Renaissance}
\begin{columns}
	\begin{column}{0.40\linewidth}
		\centering
		\includegraphics[height=0.35\paperheight]{../resources/illustrations/lunar-distance} \\
	\end{column}
	\begin{column}{0.09\linewidth} \centering \huge \emph vs. \end{column}
	\begin{column}{0.40\linewidth}
		\centering
		\includegraphics[height=0.35\paperheight]{../resources/illustrations/harrison} \\
	\end{column}
\end{columns}
\begin{columns}
	\begin{column}{0.49\linewidth}
		\begin{itemize}
			\item Logarithmes de Neper (1614)
			\item Tables de Briggs (1617)
			\item Distance lunaire (1674)
		\end{itemize}
		$\implies$ \textbf{Calculateur Humaine.}
	\end{column}
	\begin{column}{0.49\linewidth}
		\begin{itemize}
			\item Logitude Act (1614) 
			\item Horloges maritimes de Harrison (1736)
			\item Prime de £8750 (1773)
		\end{itemize}
		$\implies$ \textbf{Calcul automatique.}
	\end{column}
\end{columns}
\end{frame}
%--------------------------------------------------------------------------------
\begin{frame}
\centering
\huge \emph{À quoi sert donc l'arithmétique?}
\vfill
\centering
\includegraphics[height=0.5\paperheight]{../resources/illustrations/calculator} \\
\vfill
\end{frame}
%--------------------------------------------------------------------------------
% AUJOURD'HUI
%--------------------------------------------------------------------------------
\begin{frame}{Aujourd'hui}
\Large L'informatique est à la fois\ldots

\vfill

\begin{columns}
\begin{column}{0.33\linewidth}
\centering
\includegraphics[height=0.35\paperheight]{../resources/illustrations/histories-fragment} \\
\end{column}
\begin{column}{0.33\linewidth}
\centering
\includegraphics[height=0.35\paperheight]{../resources/illustrations/gutenberg} \\
\end{column}
\begin{column}{0.33\linewidth}
\centering
\includegraphics[height=0.35\paperheight]{../resources/illustrations/harrison} \\
\end{column}
\end{columns}

\vfill

\begin{columns}
\begin{column}{0.33\linewidth}
\centering
\ldots mémoire
\end{column}
\begin{column}{0.33\linewidth}
\centering
\ldots diffusion
\end{column}
\begin{column}{0.33\linewidth}
\centering
\ldots calcul
\end{column}
\end{columns}
\end{frame}
%--------------------------------------------------------------------------------
\begin{frame}{Aujourd'hui}
\Large Les mêmes problématiques se reposent\ldots
\vfill
\begin{columns}
	\begin{column}{0.40\linewidth}
		\centering
		\includegraphics[height=0.35\paperheight]{../resources/illustrations/modern_lecture} \\
	\end{column}
	\begin{column}{0.09\linewidth} \centering \huge \emph vs. \end{column}
	\begin{column}{0.40\linewidth}
		\centering
		\includegraphics[height=0.35\paperheight]{../resources/illustrations/Kelvin-Doe} \\
	\end{column}
\end{columns}
\begin{columns}
	\begin{column}{0.49\linewidth}
		\begin{itemize}
			\item mémorisation
			\item citation
			\item démonstration
		\end{itemize}
	\end{column}
	\begin{column}{0.49\linewidth}
		\begin{itemize}
			\item consultation au besoin
			\item expression individuelle
			\item bricolage
		\end{itemize}
	\end{column}
\end{columns}
\end{frame}
%--------------------------------------------------------------------------------
\begin{frame}
\vfill
\Huge 
\begin{center}
\emph{À quoi sert donc l'éducation?}
\end{center}
\vfill
\end{frame}
%--------------------------------------------------------------------------------
\begin{frame}
\vfill
\centering
\large
\begin{coolquote}[Condorcet, avril 1792 \footlineextra{\cite{condorcet_92}}]
Tant qu'il y aura des hommes qui n'obéiront pas à leur raison seule [\ldots] le genre humain n'en resterait pas moins partagé entre deux classes : celle des hommes qui raisonnent, et celle 
des hommes qui croient.
\end{coolquote}
\vfill
\end{frame}