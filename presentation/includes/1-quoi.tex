
\section{Introduction}

\begin{frame}{idée d'une instruction publique}

Révolution française

En avril 1792 :
\og{}tant qu'il y aura des hommes qui n'obéiront pas à leur raison seule, 
qui recevront leurs opinions d'une opinion étrangère, en vain toutes 
les chaînes auraient été brisées, en vain ces opinions de commandes 
seraient d'utiles vérités ; le genre humain n'en resterait pas moins 
partagé entre deux classes : celle des hommes qui raisonnent, et celle 
des hommes qui croient. Celle des maîtres et celle des esclaves\fg{}
Condorcet

\footlineextra{\cite{condorcet_92}}
\end{frame}

\begin{frame}{test}
\begin{itemize}
\item 1833 - loi Guizot obligeant les communes de plus de 500 habitants 
    à avoir une école primaire de garçons
\item 1836 - loi Pelet insitant les communes à avoir une école primaire de filles
\item 1867 - loi Duruy obligeant les communes de plus de 500 habitants 
à avoir une école primaire de filles
\item 1881 - loi Jules Ferry établissant la gratuité de l'enseignement primaire
\item 1882 - loi Jules Ferry rendant l'école obligatoire pour les enfants de 6 à 13 ans
\end{itemize}

\footlineextra{\cite{book_serusclat}\cite{senat_ferry}}
\end{frame}

\begin{frame}{1881 Bataillon scolaire}
\includegraphicsabsolute{../resources/illustrations/Le-bataillon-scolaire}{.5\textwidth}{1cm}{2cm}
\includegraphicsabsolute{../resources/illustrations/theorie_militaire_zoom}{.4\textwidth}{7cm}{2cm}
\end{frame}

\begin{frame}{test}
  \begin{itemize}
  \item 1890 - Questions pédagogiques, Brouard et Defondon
  \item 1899 - \og{}The child is already intensely active, and the question 
  of eductation is the question of taking hold of his activities, of 
  giving them direction\fg{} John Dewey
  \item 1923 - Constructivisme, Piaget
  \end{itemize}
  
\footlineextra{\cite{book_middle_works_dewey}\cite{piaget_1923}}
\end{frame}

\section{Intégration des TIC\ldots}

\subsection{\ldots dans l'éducation}

\begin{frame}{test}

Avec l'arrivé de l'informatique :
\begin{itemize}
\item envi de repenser l'éducation grâce à l'outil informatique (Seymour Paper Logo)
\item désillusion, intégration de l'informatique au programme scolaire comme une matière étudier de manière classique
\end{itemize}

\end{frame}



\subsection{\ldots dans la société}

\begin{frame}{test}

Tendance à déléguer des fonctionnalitées de son cerveau 
\begin{itemize}
\item mémoire
\item calcul mentale
\end{itemize}

\end{frame}
q