% !TEX root = ../apprentissage.tex

\section{Informatique, éducation et société}

%--------------------------------------------------------------------------------
% CONTEXTE HISTORIQUE
%--------------------------------------------------------------------------------
\subsection{Contexte historique}
%--------------------------------------------------------------------------------
\begin{frame}{À propos de \ldots}
\begin{coolquote}
Elle ne peut produire dans les âmes, en effet, que l’oubli de ce qu’elles savent en leur faisant négliger la mémoire [\ldots] Tu as trouvé le moyen, non point d’enrichir la mémoire, mais de conserver les souvenirs qu’elle a.
\end{coolquote}
\end{frame}
%--------------------------------------------------------------------------------
\begin{frame}{À propos de l'écriture}
\begin{coolquote}[Socrates / Platon, Phaedrus 370BC \footlineextra{\cite{plato-phaedrus}}]
Elle ne peut produire dans les âmes, en effet, que l’oubli de ce qu’elles savent en leur faisant négliger la mémoire [\ldots] Tu as trouvé le moyen, non point d’enrichir la mémoire, mais de conserver les souvenirs qu’elle a.
\end{coolquote}
\end{frame}
%--------------------------------------------------------------------------------
% PLATO VS. WRITING
%--------------------------------------------------------------------------------
\begin{frame}{Antiquité}
\begin{columns}
	\begin{column}{0.40\linewidth}
		\centering
		\includegraphics[height=0.35\paperheight]{../resources/illustrations/plato} \\
	\end{column}
	\begin{column}{0.09\linewidth} \centering \huge \emph vs. \end{column}
	\begin{column}{0.40\linewidth}
		\centering
		\includegraphics[height=0.35\paperheight]{../resources/illustrations/histories-fragment} \\
	\end{column}
\end{columns}
\begin{columns}
	\begin{column}{0.49\linewidth}
		\begin{itemize}
			\item Traditions orales~: épopées\footlineextra{\cite{havelock-preface-plato}}
			\item Dialectique / Maïeutique / Rhétorique
		\end{itemize}
		$\implies$ \textbf{Mémorisation, assimilation, débat.}\footlineextra{\cite{plato-phaedrus}}
	\end{column}
	\begin{column}{0.49\linewidth}
		\begin{itemize}
			\item Invention de l'écriture ($\approx 3500BC$)
			\item La Bible~: parole de Dieu
		\end{itemize}
		$\implies$ \textbf{Mémoire externe, consultation.}
	\end{column}
\end{columns}
\end{frame}
%--------------------------------------------------------------------------------
\begin{frame}

\centering
\Huge \emph{À quoi sert donc le par-c\oe{}ur?}
\vfill
\centering
\includegraphics[height=0.5\paperheight]{../resources/illustrations/bloom} \\
\vfill \hfill
\large Taxonomie de Bloom (1956)\footlineextra{\cite{tax-bloom}} \hfill \hfill
\end{frame}

%--------------------------------------------------------------------------------
% MONKS VERSUS PRINTING
%--------------------------------------------------------------------------------
\begin{frame}{Moyen âge}
\begin{columns}
	\begin{column}{0.40\linewidth}
		\centering
		\includegraphics[height=0.35\paperheight]{../resources/illustrations/lecture} \\
	\end{column}
	\begin{column}{0.09\linewidth} \centering \huge \emph vs. \end{column}
	\begin{column}{0.40\linewidth}
		\centering
		\includegraphics[height=0.35\paperheight]{../resources/illustrations/gutenberg} \\
	\end{column}
\end{columns}
\begin{columns}
	\begin{column}{0.49\linewidth}
		\begin{itemize}
			\item Cours magistraux
			\item Amanuensis (moines copistes)
		\end{itemize}
		$\implies$ \textbf{Prise de notes, vérité unique.}
	\end{column}
	\begin{column}{0.49\linewidth}
		\begin{itemize}
			\item Jikjii au Corée (1377)
			\item Bible de Gutenberg en Europe (1440)
		\end{itemize}
		$\implies$ \textbf{Copiage mécanique, diffusion de idées.}
	\end{column}
\footlineextra{\cite{walsham2003}}
\footlineextra{\cite{nodier1989}}
\end{columns}
\end{frame}
%--------------------------------------------------------------------------------
\begin{frame}
\centering
\huge \emph{À quoi sert donc la prise de notes?}
\vfill
\centering
\includegraphics[height=0.5\paperheight]{../resources/illustrations/cours_magistral} \\
\vfill \hfill
\large Dessin de Francis Haraux (2007).\hfill \hfill
\end{frame}

%--------------------------------------------------------------------------------
% LUNAR DISTANCE VERSUS HARRISON
%--------------------------------------------------------------------------------
\begin{frame}{À propos de\ldots}
\begin{coolquote}
Un artifice admirable qui, en réduisant à quelques jours un travail de plusieurs mois, double la durée de vie [\ldots] épargne les erreurs et le dégoût : plaies inséparables des longs calculs.
\end{coolquote}
\end{frame}
%--------------------------------------------------------------------------------
\begin{frame}{À propos des logarithmes}
\begin{coolquote}[Laplace, 1614 \footlineextra{\cite{history-of-astronomy}}]
Un artifice admirable qui, en réduisant à quelques jours un travail de plusieurs mois, double la durée de vie de l'astronome, et lui épargne les erreurs et le dégoût : plaies inséparables des longs calculs.
\end{coolquote}
\end{frame}
%--------------------------------------------------------------------------------
\begin{frame}{Renaissance}
\begin{columns}
	\begin{column}{0.40\linewidth}
		\centering
		\includegraphics[height=0.35\paperheight]{../resources/illustrations/lunar-distance} \\
	\end{column}
	\begin{column}{0.09\linewidth} \centering \huge \emph vs. \end{column}
	\begin{column}{0.40\linewidth}
		\centering
		\includegraphics[height=0.35\paperheight]{../resources/illustrations/harrison} \\
	\end{column}
\end{columns}
\begin{columns}
	\begin{column}{0.49\linewidth}
		\begin{itemize}
			\item Logarithmes de Neper (1614)
			\item Tables de Briggs (1617)
			\item Distance lunaire (1674)
		\end{itemize}
		$\implies$ \textbf{Calculateur Humaine.}
	\end{column}
	\begin{column}{0.49\linewidth}
		\begin{itemize}
			\item Pascaline (1642)
			\item Horloges maritimes de Harrison (1736)
		\end{itemize}
		$\implies$ \textbf{Calcul automatique.}
	\end{column}
\end{columns}
\end{frame}
%--------------------------------------------------------------------------------
\begin{frame}
\centering
\huge \emph{À quoi sert donc l'arithmétique?}
\vfill
\centering
\includegraphics[height=0.5\paperheight]{../resources/illustrations/calculator} \\
\vfill
\end{frame}
%--------------------------------------------------------------------------------
% AUJOURD'HUI
%--------------------------------------------------------------------------------
\begin{frame}{Aujourd'hui}
\Large L'informatique est à la fois\ldots

\vfill

\begin{columns}
\begin{column}{0.33\linewidth}
\centering
\includegraphics[height=0.35\paperheight]{../resources/illustrations/histories-fragment} \\
\end{column}
\begin{column}{0.33\linewidth}
\centering
\includegraphics[height=0.35\paperheight]{../resources/illustrations/gutenberg} \\
\end{column}
\begin{column}{0.33\linewidth}
\centering
\includegraphics[height=0.35\paperheight]{../resources/illustrations/harrison} \\
\end{column}
\end{columns}

\vfill

\begin{columns}
\begin{column}{0.33\linewidth}
\centering
\ldots mémoire
\end{column}
\begin{column}{0.33\linewidth}
\centering
\ldots diffusion
\end{column}
\begin{column}{0.33\linewidth}
\centering
\ldots calcul
\end{column}
\end{columns}
\end{frame}
%--------------------------------------------------------------------------------
\begin{frame}
\centering
\huge \emph{À quoi sert donc l'éducation?}
\vfill
\centering
\large
\begin{coolquote}[Condorcet, avril 1792 \footlineextra{\cite{condorcet_92}}]
Tant qu'il y aura des hommes qui n'obéiront pas à leur raison seule [\ldots] le genre humain n'en resterait pas moins partagé entre deux classes : celle des hommes qui raisonnent, et celle 
des hommes qui croient.
\end{coolquote}
\vfill
\end{frame}


%--------------------------------------------------------------------------------
% ÉPOQUE CONTEMPORAINE
%--------------------------------------------------------------------------------


\subsection{Époque contemporaine}

\begin{frame}{Idée d'une instruction publique}

Révolution française

\begin{coolquote}[Condorcet, avril 1792]
Tant qu'il y aura des hommes qui n'obéiront pas à leur raison seule [\ldots] le genre humain n'en resterait pas moins partagé entre deux classes : celle des hommes qui raisonnent, et celle 
des hommes qui croient.
\end{coolquote}

\footlineextra{\cite{condorcet_92}}
\end{frame}

\begin{frame}{Bref historique de l'institution scolaire}
\begin{description}
\item[\bf 1833 - loi Guizot] obligeant les communes de plus de 500 habitants à avoir une école primaire de garçons
\item[\bf 1836 - loi Pelet] insitant les communes à avoir une école primaire de filles
\item[\bf 1867 - loi Duruy] obligeant les communes de plus de 500 habitants à avoir une école primaire de filles
\item[\bf 1881 - loi Jules Ferry] établissant la gratuité de l'enseignement primaire
\item[\bf 1882 - loi Jules Ferry] rendant l'école obligatoire pour les enfants de 6 à 13 ans
\end{description}

\footlineextra{\cite{book_serusclat}\cite{senat_ferry}}
\end{frame}

\begin{frame}{1881 Bataillon scolaire}
\includegraphicsabsolute{../resources/illustrations/Le-bataillon-scolaire}{.5\textwidth}{1cm}{2cm}
\includegraphicsabsolute{../resources/illustrations/theorie_militaire_zoom}{.4\textwidth}{7cm}{2cm}
\end{frame}

\begin{frame}
  \begin{itemize}
  \item 1890 - Questions pédagogiques, Brouard et Defondon
  \item 1899 - \og{}The child is already intensely active, and the question 
  of eductation is the question of taking hold of his activities, of 
  giving them direction\fg{} John Dewey
  \item 1923 - Constructivisme, Piaget
  \end{itemize}
  
\footlineextra{\cite{book_middle_works_dewey}\cite{piaget_1923}}
\end{frame}


