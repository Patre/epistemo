% !TEX root = ../apprentissage.tex

\section{TIC, éducation et société}


%--------------------------------------------------------------------------------
% CONTEXTE HISTORIQUE
%--------------------------------------------------------------------------------
\subsection{Époque ancienne}

%--------------------------------------------------------------------------------
% PLATO VS. WRITING
%--------------------------------------------------------------------------------
\begin{frame}{Antiquité}
\begin{columns}
	\begin{column}{0.40\linewidth}
		\centering
		\includegraphics[height=0.35\paperheight]{../resources/illustrations/plato} \\
	\end{column}
	\begin{column}{0.09\linewidth} \centering \huge \emph vs. \end{column}
	\begin{column}{0.40\linewidth}
		\centering
		\includegraphics[height=0.35\paperheight]{../resources/illustrations/histories-fragment} \\
	\end{column}
\end{columns}
\begin{columns}
	\begin{column}{0.49\linewidth}
		\begin{itemize}
			\item Traditions orales~: épopées
			\item Dialectique / Maïeutique
		\end{itemize}
		$\implies$ \textbf{Mémorisation, assimilation.}
	\end{column}
	\begin{column}{0.49\linewidth}
		\begin{itemize}
			\item Invention de l'écriture ($\approx -3500$)
		\end{itemize}
		$\implies$ \textbf{Mémoire externe, conaissance superficielle}
	\end{column}
\end{columns}
\end{frame}

%--------------------------------------------------------------------------------
% MONKS VERSUS PRINTING
%--------------------------------------------------------------------------------
\begin{frame}{Moyen âge}
\begin{columns}
	\begin{column}{0.40\linewidth}
		\centering
		\includegraphics[height=0.35\paperheight]{../resources/illustrations/lecture} \\
	\end{column}
	\begin{column}{0.09\linewidth} \centering \huge \emph vs. \end{column}
	\begin{column}{0.40\linewidth}
		\centering
		\includegraphics[height=0.35\paperheight]{../resources/illustrations/gutenberg} \\
	\end{column}
\end{columns}
\begin{columns}
	\begin{column}{0.49\linewidth}
		\begin{itemize}
			\item Cours magistraux
			\item Moines copistes
		\end{itemize}
		$\implies$ \textbf{Prise de notes.}
	\end{column}
	\begin{column}{0.49\linewidth}
		\begin{itemize}
			\item Jikjii au Corée (1377)
			\item Bible de Gutenberg en Europe (1440)
		\end{itemize}
		$\implies$ \textbf{Copillage mécanique.}
	\end{column}
\end{columns}
\end{frame}

%--------------------------------------------------------------------------------
% LUNAR DISTANCE VERSUS HARRISON
%--------------------------------------------------------------------------------
\begin{frame}{Renaissance}
\begin{columns}
	\begin{column}{0.40\linewidth}
		\centering
		\includegraphics[height=0.35\paperheight]{../resources/illustrations/lunar-distance} \\
	\end{column}
	\begin{column}{0.09\linewidth} \centering \huge \emph vs. \end{column}
	\begin{column}{0.40\linewidth}
		\centering
		\includegraphics[height=0.35\paperheight]{../resources/illustrations/harrison} \\
	\end{column}
\end{columns}
\begin{columns}
	\begin{column}{0.49\linewidth}
		\begin{itemize}
			\item Logarithmes de Neper (1614)
			\item Distance lunaire (1674)
		\end{itemize}
		$\implies$ \textbf{Algorithmes, pré-calcul.}
	\end{column}
	\begin{column}{0.49\linewidth}
		\begin{itemize}
			\item Pascaline (1642)
			\item Horloges maritimes de Harrison (1736)
		\end{itemize}
		$\implies$ \textbf{Calcul automatique.}
	\end{column}
\end{columns}
\end{frame}

\subsection{Époque contemporaine}

\begin{frame}{Idée d'une instruction publique}

Révolution française

\begin{quote}
Tant qu'il y aura des hommes qui n'obéiront pas à leur raison seule [\ldots] le genre humain n'en resterait pas moins partagé entre deux classes : celle des hommes qui raisonnent, et celle 
des hommes qui croient.
\end{quote}
Condorcet, avril 1792

\footlineextra{\cite{condorcet_92}}
\end{frame}

\begin{frame}{Bref historique de l'institution scolaire}
\begin{description}
\item[\bf 1833 - loi Guizot] obligeant les communes de plus de 500 habitants à avoir une école primaire de garçons
\item[\bf 1836 - loi Pelet] insitant les communes à avoir une école primaire de filles
\item[\bf 1867 - loi Duruy] obligeant les communes de plus de 500 habitants à avoir une école primaire de filles
\item[\bf 1881 - loi Jules Ferry] établissant la gratuité de l'enseignement primaire
\item[\bf 1882 - loi Jules Ferry] rendant l'école obligatoire pour les enfants de 6 à 13 ans
\end{description}

\footlineextra{\cite{book_serusclat}\cite{senat_ferry}}
\end{frame}

\begin{frame}{1881 Bataillon scolaire}
\includegraphicsabsolute{../resources/illustrations/Le-bataillon-scolaire}{.5\textwidth}{1cm}{2cm}
\includegraphicsabsolute{../resources/illustrations/theorie_militaire_zoom}{.4\textwidth}{7cm}{2cm}
\end{frame}

\begin{frame}{test}
  \begin{itemize}
  \item 1890 - Questions pédagogiques, Brouard et Defondon
  \item 1899 - \og{}The child is already intensely active, and the question 
  of eductation is the question of taking hold of his activities, of 
  giving them direction\fg{} John Dewey
  \item 1923 - Constructivisme, Piaget
  \end{itemize}
  
\footlineextra{\cite{book_middle_works_dewey}\cite{piaget_1923}}
\end{frame}


