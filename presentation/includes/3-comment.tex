\section{Initiatives \& Solutions}

\subsection{Pouvoirs publics}

\begin{frame}{test}
test
\end{frame}

\subsection{Autres acteurs}

\begin{frame}{Moodle}
\end{frame}

\begin{frame}{One Laptop Per Child}
  \begin{itemize}
   \item Nicholas  Negroponte, fondateur du MediaLab (MIT)
   \item Inspiration du Dynabook d'Alan Kay
  \end{itemize}
\end{frame}

\begin{frame}{Hole in the Wall}
  \begin{itemize}
    \item Sugata Mitra  
  \end{itemize}
\end{frame}

\subsection{Quelles solutions ?}

\begin{frame}{S'appuyer sur le constructi[visme|onisme] ?}
\begin{itemize}
  \item \textbf{Constructivisme 1923 - Piaget}. On suppose que les individus ne perçoivent pas une copie de la réalité qui les entours, mais une reconstruction interne. L'individu reconstruit en permanance les objets perçus en fonction des concept déjà intégrés.
  \item \textbf{Constructionnisme 1960 - Seymour Papert}. En accord avec le constructivisme mais met en évidence l'importance de la construction d'un "entité publique" comme le dialogue et l'interaction.
\end{itemize}
\end{frame}

\begin{frame}{S'appuyer sur le constructi[visme|onisme] ?}
\begin{itemize}
  \item Une personne motivée réussira à apprendre par elle même
  \item Nécessité d'interactions sociales (dialogues, débats) pour l'assimilation (la CONSTRUCTION) des nouveaux concepts
\end{itemize}
\end{frame}