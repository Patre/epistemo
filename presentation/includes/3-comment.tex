\section{Initiatives \& Solutions}

\subsection{Pouvoirs publics}

\begin{frame}{test}
test
\end{frame}

\subsection{Autres acteurs}

\begin{frame}{Moodle}
\end{frame}

\begin{frame}{One Laptop Per Child}
  \begin{itemize}
   \item Nicholas  Negroponte, fondateur du MediaLab (MIT)
   \item Inspiration du Dynabook d'Alan Kay
   \item 1986 (présenté en 1978) : Seymour Papert à fait l'observation suivante [Teaching childs thinking \footnote{\texttt{http://stager.org/articles/teachingchildrenteaching.pdf}}] : les enfants qui écrivent des programmes informatique comprennent les choses différemment, et quand ils les debug, ils arrivent presque à apprendre sur l'apprentissage.
   \item 1982 : travail avec Seymour Papert : les premier à apporter des ordinateurs aux écoles des pays en voie de développement
  \end{itemize}
  Souhait de Nicholas de fonder un projet important lors de son départ du Medialab : OLPC. L'idée est de s'occuper d'éducation en agissant sur les enfants :
  \begin{itemize}
  \item en leur apportant l'accès aux ordinateurs,
  \item 
  \end{itemize}
  -> Les enfant y arrivent aussi bien qu'ici
\end{frame}

\begin{frame}{Hole in the Wall}
  \begin{itemize}
    \item Sugata Mitra  
  \end{itemize}
\end{frame}

\subsection{Quelles solutions ?}

\begin{frame}{S'appuyer sur le constructi[visme|onisme] ?}
\begin{itemize}
  \item \textbf{Constructivisme 1923 - Piaget}. On suppose que les individus ne perçoivent pas une copie de la réalité qui les entours, mais une reconstruction interne. L'individu reconstruit en permanance les objets perçus en fonction des concept déjà intégrés.
  \item \textbf{Constructionnisme 1960 - Seymour Papert}. En accord avec le constructivisme mais met en évidence l'importance de la construction d'un "entité publique" comme le dialogue et l'interaction.
\end{itemize}
\end{frame}

\begin{frame}{S'appuyer sur le constructi[visme|onisme] ?}
\begin{itemize}
  \item Une personne motivée réussira à apprendre par elle même
  \item Nécessité d'interactions sociales (dialogues, débats) pour l'assimilation (la CONSTRUCTION) des nouveaux concepts
\end{itemize}
\end{frame}

\begin{frame}{Concrètement}
\begin{itemize}
  \item D'abord confronter l'élève à la problématique avant de lui enseigner les concepts abstraits
\end{itemize}
\end{frame}