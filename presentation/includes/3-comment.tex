\section{Initiatives \& Solutions}

\subsection{Pouvoirs publics}

\begin{frame}{test}
test
\end{frame}

\subsection{Autres acteurs}

\begin{frame}{Moodle}
\end{frame}

\begin{frame}{One Laptop Per Child}
  \begin{block}{Nicholas Negroponte}
    \begin{itemize}
      % Passionné d'art et fort en math, il décide d'intégrer le MIT en architecture (qui combine les deux domaine)
      \item Passionné d'art \& fort en Math $\rightarrow$ Architecture MIT
      % A la sortie de ses études, il s'est rendu compte que l'ordinateur permettai de combiner art et sciences en permettant aux gens d'être créatif
      \item Art \& Sciences $\rightarrow$ Ordinateur
      % Il décide de créer le MediaLab au sein du MIT
      \item MediaLab au MIT (1985)
      % Puis il quitte le poste de président du MediaLab dans les année 2000 avec la volonté de mener un projet important
      \item Projet OLPC (2005)
    \end{itemize}
  \end{block}
\end{frame}

\begin{frame}{One Laptop Per Child}
  \begin{itemize}
   \item Nicholas  Negroponte, fondateur du MediaLab (MIT)
   \item Inspiration du Dynabook d'Alan Kay
   \item 1986 (présenté en 1978) : Seymour Papert à fait l'observation suivante [Teaching childs thinking \footnote{\texttt{http://stager.org/articles/teachingchildrenteaching.pdf}}] : les enfants qui écrivent des programmes informatique comprennent les choses différemment, et quand ils les debug, ils arrivent presque à apprendre sur l'apprentissage.
   \item FINALITÉ : faire programmer les enfants ! -> Squeak LOGO 
   \item 1982 : travail avec Seymour Papert : les premier à apporter des ordinateurs aux écoles des pays en voie de développement
  \end{itemize}
  Souhait de Nicholas de fonder un projet important lors de son départ du Medialab : OLPC. L'idée est de s'occuper d'éducation en agissant sur les enfants :
  \begin{itemize}
  \item en leur apportant l'accès aux ordinateurs,
  \item 
  \end{itemize}
  -> Les enfant y arrivent aussi bien qu'ici
\end{frame}

\begin{frame}{Hole in the Wall}
  \begin{itemize}
    \item Sugata Mitra  
  \end{itemize}
  \footlineextra{\cite{website_hole_in_the_wall}}
\end{frame}

\subsection{Quelles solutions ?}

\begin{frame}{À plusieurs niveaux}
\begin{block}{Enseignements}
Nécessité de former les jeunes aux nouvelles technologies, aux aspects techniques mais aussi et surtout les éduquer face aux dangers que représente la société hyper-connectée d'aujourd'hui.
\end{block}
\begin{block}{Méthodes}
Revoir les méthodes d'apprentissages afin de profiter pleinement des TIC dans nos écoles, cela exige une remise en question majeure du système et des méthodes éducatives actuels.
\end{block}
\end{frame}

\begin{frame}{Enseignements}
\Huge \textbf{:)}
\end{frame}

\begin{frame}{Méthodes}
\textbf{\Huge  S'appuyer sur le constructi[visme|onisme] ?}
\begin{itemize}
  \item \textbf{Constructivisme 1923 - Piaget}. On suppose que les individus ne perçoivent pas une copie de la réalité qui les entours, mais une reconstruction interne. L'individu reconstruit en permanance les objets perçus en fonction des concept déjà intégrés.
  \item \textbf{Constructionnisme 1960 - Seymour Papert}. En accord avec le constructivisme mais met en évidence l'importance de la construction d'une "entité publique" comme le dialogue et l'interaction. Rôle des constructions réelles dans les constructions mentales.
\end{itemize}
\end{frame}

\begin{frame}{Méthodes}
\begin{itemize}
  \item Une personne motivée réussira à apprendre par elle même
  \item Nécessité de constructions réelles pour l'assimilation (la construction / amélioration) des nouveaux concepts mentaux
  \item Idée de bricoler les concept. Terme introduit Claude Lévi-Strauss dans \textit{La pensée sauvage}. % Le bricolage s'oppose à la pensée unique (méthode analytique) instruite dans le système éducatif traditionnel. Idée de bricolage (terme repris de Lévi-Strauss, Pensées sauvages) est de travailler sur du concret afin d'arriver à un résultat de manière détournée (comme le ferait un bon bricoleur).
  \item L'ordinateur donne de nombreuses opportunités de bricolages.
  \item Le constructionnisme s'oppose à l'insctructionnisme.
  \begin{description}
    \item[Instructionnisme : ]affirme la croyance que pour améliorer l'apprentissage il faut développer l'instruction (enseigner mieux).
    \item[Constructionnisme : ]favoriser le plus grand apprentissage avec le moins d'enseignement possible.
    \end{description}
    \og{}Le savoir dont les enfants ont le plus besoin est celui qui leur permet d'en acquérir davantage\fg (Papert p.141).
\end{itemize}
\end{frame}

\begin{frame}{Concrètement}
\begin{itemize}
  \item D'abord confronter l'élève à la problématique avant de lui enseigner les concepts abstraits
  \item Réaliser des projets plutôt que résoudre des problèmes
  \item 
\end{itemize}
\end{frame}