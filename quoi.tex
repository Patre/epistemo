% !TEX encoding = UTF-8 Unicode
% !TEX root = rapport.tex

\chapter{Historique de l'intégration des technologies de l'information \ldots}\label{quoi}

\section{\ldots dans l'éducation}

Apparu vers 3500 années avant JC, l'écriture est probablement la première 
technolgie de l'information. Cette invention fut critiqué notamment par Platon
à travers un dialogue entre Socrates et Phaedrus. Pour lui se reposer sur 
écriture nuit à la mémoire. Ironiquement si nous conaissons Platon c'est grace
à des records écrites. Cf. aussi le 7ème Epitre de Platon.

"Elle ne peut produire dans les âmes, en effet, que l’oubli de ce qu’elles 
savent en leur faisant négliger la mémoire. Parce qu’ils auront foi dans 
l’écriture, c’est par le dehors, par des empreintes étrangères, et non plus du 
dedans et du fond d’eux-mêmes, que les hommes chercheront à se ressouvenir. Tu 
as trouvé le moyen, non point d’enrichir la mémoire, mais de conserver les 
souvenirs qu’elle a. Tu donnes à tes disciples la présomption qu’ils ont la 
science, non la science elle-même. Quand ils auront, en effet, beaucoup appris 
sans maître, ils s’imagineront devenus très savants, et ils ne seront pour la 
plupart que des ignorants de commerce incommode, des savants imaginaires au lieu
 de vrais savants. "
\url{https://cercamon.wordpress.com/2006/05/10/platon-sur-lecriture-phedre-274-276/}
(Plato, Phaedrus 275a-b)

Les logarithms de Napier, introduites en 1614 dans son oeuvre "Mirifici 
Logarithmorum Canonis Descriptio", furent un outil de calcul abstrait visant à 
simplifier les calcules longues. À l'époque les tables de logarithms (Henry 
Briggs 1617) apprises par coeur [citation needed] étaient indispensables tout 
mathématicien, ingénieur ou scientifique. De nos jours ces tables ne sont mêmes 
pas introduites à l'école. 

Des débats intenses suivent la popularisation de la calculatrice éléctronique 
aux années 1970: suit un débat. Il est important de se rappeller que les 
machines à calculer tels l'Abacus existent depuis environ 4000 ans. Ceux-ci 
portent cependant certaines avantages pédaogiques [citation needed]: c'est 
l'usager qui déroule l'algorithme et non la machine qui le fait tout seul. 

Depuis les années 90: internet, outil pour tricher?
\url{http://www.apsq.org/sautquantique/doss/d-tricherie.html/}

\ldots

\section{\ldots dans la société}

La notion de nation a t'il de pertienence dans un monde connecté par internet?

"The Big Switch: Rewiring the World from Edison to Google" 
 \url{http://www.nicholasgcarr.com/bigswitch/}

Nous perdones notre confidentialité de l'intime. Pour Google, Facebook, etc, 
c'est pas un problème: ceux qui n'ont rien à cacher n'ont pas à ce soucier. Nous
pouvion cependant noter qu'ils ont intéret de penser ainsi [j'arrive pas à 
trouver l'article qui en parle]. Le voyeurisme de la télévision réalité montre 
que ce phénomène est de Zeitgeist (ésprite de l'époque): nous nous soucions 
moins, par exemple, de qui nous entend parler sur nos téléphones portables.

Les réseaux sociales permettent un exhibitionisme, et ceci conduit au 
narcissime:
http://www.guardian.co.uk/technology/2012/mar/17/facebook-dark-side-study-aggressive-narcissism 
[WHY U NO GIVE ME LINK TO ORIGINAL ARTICLE!]

"The Spy in the Coffee Machine: The End of Privacy As We Know It"
 \url{http://eprints.soton.ac.uk/265683/}

En Fevrier 2010 Eben Moglen lance la notion de "Freedom Box" pour faire face à
la centralisation progressive du Web. C'est d'ailleurs sa présentation qui fut
l'ammorce du projet Diaspora, visant à créer une platforme sociale décentralisé 
respectant la confidentialité de ses usagers. 

\subsection{\ldots par les individus}


