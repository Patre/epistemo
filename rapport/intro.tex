% !TEX encoding = UTF-8 Unicode
% !TEX root = rapport.tex

\chapter*{Introduction}\label{intro}

\begin{coolquote}[Albert Einstein (attribué)]
L'éducation est ce qui reste après avoir oublié tout ce que nous avons appris à l'école. 
\end{coolquote}

Il nous semble qu'aujourd'hui l'éducation a deux objectifs~: le premier, très pragmatique, est d’amorcer les carrières des futurs travailleurs en leur donnant les outils dont ils auront besoin dans le monde du travail. Le second, moins tangible et peut-être en voie de disparition, est de permettre au futurs citoyens un épanouissement de leurs esprits afin de favoriser leurs capacités à \og{}faire de bons choix\fg{}.

Il est fréquent d'entendre des élèves se plaindre des moyens d'évaluation dans le système éducatif. Voici un exemple concret d'un étudiant dans notre formation déplorant le contenu d'un examen de réseaux. Parmi les questions posées figurait la suivante (aucun document autorisé)~:

\begin{coolquote}[Étudiant en Licence Informatique]Quelle est la
  longueur en bits de l'entête d'un paquet IPv4~?\end{coolquote}

Cette question incarne un réel problème auquel nous faisons face
aujourd'hui, qui amène à la question suivante~: quelle est l'utilité
pour un étudiant de connaître un cours par c\oe{}ur~? Que ce soit du
point de vue économique ou philosophique, nous pensons que cette
utilité est faible.

Ce constat nous a amené à nous poser de nombreuses questions par rapport à la mémorisation, la prise de notes, le calcul de tête, etc. Sont-elles de vraies compétences que nos anciens essaient de nous transférer~? Ou, au contraire, de simples outils pour \textbf{écrémer} chaque génération de ses intellectuelles~? L'éducation n'est-elle donc qu'un simple jeu, dont la porté ne s'étend pas au delà des portes de ses établissements ? Nous pensons évidemment que la réponse est non.

Pourtant, bien qu'à une époque ces méthodes d'éducation étaient conformes aux besoins de la société et des individus, avec l'arrivée de l'outil informatique, elles semblent l'être de moins en moins. La société change, les individus changent, et l'éducation ne change pas, ou pas assez rapidement. Mais doit-elle vraiment changer~? Et si oui, comment~?

Dans un premier temps, un historique de l'intégration des technologies informatiques dans l'éducation et dans la société sera établi, dans le but de définir le contexte dans lequel nous nous plaçons. Arrivés à l'époque moderne nous dériverons plus formellement la problématique centrale sur laquelle nous nous concentrerons~: le décalage progressif qui se dégage entre les attentes de la société \textbf{et ce que fournit l'éducation au futurs citoyens}. Ensuite nous exposerons un ensemble d'initiatives mises en place par différents acteurs dans le but de réduire ce décalage. Enfin, nous mettrons en évidences les points essentiels qu'il serait simple de mettre en place dans le système actuel.

\vfill
\section*{Mots clés}
Éducation -- Informatique -- Système éducatif -- Méthode d'apprentissage -- Constructivisme