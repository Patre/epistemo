% !TEX encoding = UTF-8 Unicode
% !TEX root = rapport.tex

\chapter*{Introduction}\label{intro}

\begin{coolquote}[Albert Einstein (attribué)]
\Large L'éducation est ce qui reste après avoir oublié tout ce que nous avons appris à l'école. 
\end{coolquote}

\textit{Il me semble qu'aujourd'hui l'éducation a deux objectives~: la première, très utilitaire, est d’amorcer les carrières des futurs travailleurs en leur donnant les outils dont ils auront besoin dans le monde du travail. Le second, moins tangible et peut-être en voie de disparition, est de faire épanouir les esprits des futurs citoyens et êtres humains pour qu'ils puissent faire les bons choix pour leur pays et leur planète.}

\textit{Hier un ami de Licence est venu me voir pour se plaindre d'un examen de réseaux. Parmi les questions posés se figurait la suivante, à réciter de tête~:}

\begin{coolquote}\Large Quelle est la longueur en bits de l'entête d'un paquet IPv4?\end{coolquote}

\textit{Pour moi cette question incarne un problème auquel nous faisons face aujourd'hui~: à mon avis il ne sert nullement à l'étudiant de connaître la réponse par c\oe{}ur, que ce soit du point de vue économique ou philosophique. L'éducation, il me semble, ne pose pas les bonnes questions.}

\textit{Ce constat nous a amené à se questionner par rapport à la mémorisation, la prise de notes, le calcul de tête,\ldots sont-elles de vraies compétences que nos anciens essaient de nous transférer, ou des simples outils pour écrémer chaque génération de ses intellectuelles? L'éducation n'est-ce donc qu'un simple jeu, dont la porté ne s'étend pas plus loin que les portes de ses établissements? La réponse, surement, devrait être non.}

\textit{Pourtant si à une époque l'éducation était conforme au besoins de la société et des individus, avec l'arrivée de l'outil informatique il semble l'être de moins en moins. La société change, les individus changent, et l'éducation ne change pas, ou pas assez rapidement. Mais doit-il vraiment changer? Et si oui, comment?}

\textit{Nous commencerons ce rapport en vous présentant un historique de l'intégration des technologies informatiques dans l'éducation et dans la société, à fin de poser un contexte pour ce qui suit. Arrivés à l'époque moderne nous dériverons plus formellement la problématique centrale qui nous intéresse~: celui du décalage progressive qui se dégage entre les attentes de la société et ce qui fournit l'éducation au futurs citoyens. Ensuite nous évaluerons un ensemble d'initiatives mises en place pour faire face à ce décalage, pour finir avec nos humbles propositions.}