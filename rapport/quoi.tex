% !TEX encoding = UTF-8 Unicode
% !TEX root = rapport.tex

\part{Historique de l'intégration des technologies de l'information \ldots}\label{quoi}

\chapter*{Avant-propos~: Technologies Informatiques?}

\begin{minipage}[H]{0.3\linewidth}
  \begin{figure}[H]
  \centering
  \includegraphics[width=0.8\textwidth]{../resources/illustrations/steinbuch}
  \caption{Karl Steinbuch}
  \end{figure}
\end{minipage}
\begin{minipage}[H]{0.7\linewidth}
Le mot \textbf{informatique} est une concatenation d'\textbf{information} et d'\textbf{automatique} fait en 1957 par Karl Steinbuch\cite{steinbuch-2005} pour décrire le traitement automatique de l'information.

Depuis le terme a été adopté pour décrire une gamme tellement vaste de sciences, de technologies et de services qu'il a besoin d'être qualifié pour avoir un sens précis.
\vspace{1cm}
\end{minipage}

\begin{coolquote}[Wikipedia\cite{wiki-informatique}]
Les expressions \textbf{science informatique}, \textbf{informatique fondamentale} ou \textbf{informatique théorique} sont utilisées pour désigner sans ambiguïté la science, tandis que \emph{technologies de l'information} ou \textbf{technologies de l'information et de la communication} désignent le secteur industriel et ses produits.
\end{coolquote}

Ici nous utiliserons \textbf{technologies de l'information} pour décrire toute technique permettant de stoquer, de traiter ou de transférer l'information. 

Cette définition couvre évidemment les \emph{algorithmes}, \emph{structure données} et \emph{protocoles de communication}, mais aussi les \emph{langages}, naturels ou non, l'\emph{écriture} et tout outil de réflexion, de stockage, transformation ou de transfert de l'information\ldots

\chapter{Le langage~: protocole de communication}
Avec l'apparition du langage l'Être humain s'est doté d'un outil capable d'exprimer des informations de plus en plus complexes. Environ 6000 langues sont parlés aujourd'hui; les Anthropologues n'ont jamais découvert un peuple démunie d'un langage complexe\cite{linguistics-pinker}. 

Il est difficile de dater l'appararition du langage. L'Homme essaye pourtant depuis des milliers d'années de trouver un \emph{proto-language}, origine de toute autre, mais sans succès. Dans \textbf{L'Enquête}, l'historien Grecque Hérodote fait référence à une expérience de \emph{privation de langage} du pharaon Psammétique~I visant à identifier une sorte de langage \emph{par défaut} de l'Homme\cite{herodote-privation}. 

\begin{minipage}[H]{0.32\linewidth}
  \begin{figure}[H]
  \centering
  \includegraphics[width=0.8\textwidth]{../resources/illustrations/psamtik-I}
  \caption{Pharaon Psammétique~I}
  \end{figure}
\end{minipage}
\begin{minipage}[H]{0.32\linewidth}
  \begin{figure}[H]
  \centering
  \includegraphics[width=0.8\textwidth]{../resources/illustrations/herodote}
  \caption{Historien Hérodote}
  \end{figure}
\end{minipage}
\begin{minipage}[H]{0.32\linewidth}
  \begin{figure}[H]
  \centering
  \includegraphics[width=0.8\textwidth]{../resources/illustrations/fred-II}
  \caption{Empereur Frédéric~II}
  \end{figure}
\end{minipage}

\begin{minipage}[H]{0.65\linewidth}
De tels expériences furènt répétés au fil de l'histoire, notamment par l'empereur Frédéric~II de Hohenstaufen, pour qui le résultat fut la mort des nourissons volontaires\cite{ggcoulton-francis-to-dante}. 

Des exemples plus modernes d'\emph{enfants sauvages} ont conduits à l'hypothèse de la \emph{période critique} de Wilder Penfield et Lamar Roberts\cite{penfield2003speech}, popularisé par Eric Lenneberg\cite{lenneberg-crit-period}~: un enfant privé de vocalisations pendant les première années de sa vie sera incapable de bien assimiler le langage par la suite.
\vspace{1cm}
\end{minipage}
\begin{minipage}[H]{0.34\linewidth}
  \begin{figure}[H]
  \centering
  \includegraphics[width=0.8\textwidth]{../resources/illustrations/penfield}
  \caption{Wilder Penfield}
  \end{figure}
\end{minipage}

Peu importe sa ou ses origines, le langage est une technologie de l'information primordiale. Il permet un transfert d'informations entre membres d'une société donc, par le bias des traditions orales, un stockage de l'information pendant une durée plus longue que la vie d'un individu. Ainsi munis de la capacité de se propager en dehors de leurs hôtes, les mèmes deviennent alors potentiellement immortels.

Chez les peuples alettrés une deuxième invention, la structure d'épopée, est souvent utilisé pour faciliter partage et mémorisation des récits sous form de poésies rythmés \cite{havelock-preface-plato}. Notons d'ailleurs que la poésie est toujours utilisé aujourd'hui comme exercise de mémorisation (à l'école) et moyen mnémotechnique (surtout dans la publicité).

\chapter{L'écriture~: mémoire externe}
Si le langage peut être vu comme un instinct plutôt qu'une technologie, ce n'est pas le cas de l'écriture. L'écriture idéographique n'est apparu, il y environ 5000 ans, chez un nombre limité de civilisations. L'écriture \emph{alphabétique} d'ailleurs semble n'être apparu qu'une seule fois, chez les Cananéens, il y a 3700 ans\cite{linguistics-pinker}. Toute autre écriture alphabétique se serait donc dérivé de celle-ci.
%\gls{Écriture alphabétique~: écriture où chaque symbole correspond à un son vocal.}

En tout cas l'invention permet à un support mort de stoquer un ensemble de données codés sous forme de symboles, donc de repousser d'avantage la frontière de l'espace et du temps. Cette avancé rend presque redondants les techniques de mémorisation lyriques mentionnées ci-dessus. La réaction de ceux qui se sont investies dedans est donc peu surprenant~:

\begin{coolquote}[Socrates (attribué)\cite{plato-phaedrus}]
Elle ne peut produire dans les âmes, en effet, que l’oubli de ce qu’elles  savent en leur faisant négliger la mémoire. Parce qu’ils auront foi dans  l’écriture, c’est par le dehors, par des empreintes étrangères, et non plus du dedans et du fond d’eux-mêmes, que les hommes chercheront à se ressouvenir. 
\end{coolquote}

\begin{minipage}[H]{0.3\linewidth}
  \begin{figure}[H]
  \centering
  \includegraphics[width=0.8\textwidth]{../resources/illustrations/bloom_face}
  \caption{Benjamin Bloom}
  \end{figure}
\end{minipage}
\begin{minipage}[H]{0.69\linewidth}
Cette critique fait à travers un dialogue entre Socrates et Phaedrus, suggère que l'écriture, en plus de nuire à la mémoire, limite son lecteur à ce que la taxonomie de Bloom\cite{tax-bloom} appelerait \emph{connaître} à la différence de \emph{comprendre}, \emph{appliquer} et cetera. 
\vspace{1cm}
\end{minipage}

\begin{coolquote}[Socrates (attribué), -370\cite{plato-phaedrus}]
Tu as trouvé le moyen, non point d’enrichir la mémoire, mais de conserver les souvenirs qu’elle a. Tu donnes à tes disciples la présomption qu’ils ont la science, non la science elle-même. Quand ils auront, en effet, beaucoup appris sans maître, ils s’imagineront devenus très savants, et ils ne seront pour la plupart que des ignorants de commerce incommode, des savants imaginaires au lieu de vrais savants.
\end{coolquote}

\begin{minipage}[H]{0.49\linewidth}
  \begin{figure}[H]
  \centering
  \includegraphics[height=0.15\paperheight]{../resources/illustrations/plato2}
  \caption{Plato}
  \end{figure}
\end{minipage}
\begin{minipage}[H]{0.49\linewidth}
  \begin{figure}[H]
  \centering
  \includegraphics[height=0.15\paperheight]{../resources/illustrations/socrates}
  \caption{Socrates}
  \end{figure}
\end{minipage}

Pour Platon une conaissance ne peut être transféré correctement sans être assimilé, or le papier, si bien soit-il capable de stoquer des propos, est incapable de les comprendre ou de les défendre.

Il ne faut pas oublier, cependant, que Platon était un professeur dont la pédagogie se repossait sur le dialogue et non la lecture. Il n'est donc pas un interlocuteur très objective. Notons également avec ironie que si nous connaissons Platon c'est grâce à l'écriture, et si nous confondons Socrates avec lui c'est que ce dernier n'a laissé aucune trace écrite.

Celà étant, les critiques de Platon restent pertinentes aujourd'hui, voir le deviennent d'autant plus~: la consultation d'informations est maintenant tellement facile que la mémorisation semble suranné, mais il y une distinction importante à faire entre éfleurer un propos et l'assimiler.

Quand nous nous approprions véritablement d'un propos il devient bien plus qu'une copie supplémentaire redondant~: l'idée nous appartient, nous change et est changé pour nous\ldots

\chapter{L'imprimerie~: reproduction des données}

Malgré les réticences des philosophes Grecques, l'éducation en Europe pendant le Moyens Âge se repose sur l'écrit et non le discours, l'académique de l'époque se voyant comme conservateur de patrimoine Classique et non chercheur de nouvelles idées\cite{friesen-the-lecture}.

L'enseignement se fait alors par le bias de \emph{lectures}, mot qui est d'ailleurs toujours utilisé en Anglais pour dire \emph{cours magistral}~: c'est d'ailleur à ce moment que cette forme de pédagogie apparait, surtout pour faire face à la rareté de textes.

En effet à l'époque le livre est un véritable objet d'art, un trésor patrimoine faite pour survivre à des génération de bibliothécaires~: chaque ouvrage est retranscrit et décoré avec soin par un copiste artisan. Le futur savant ne pouvant pas s'offrir tous les textes dont il aura besoin par la suite doit donc les retranscire lui-même\cite{friesen-the-lecture}. Le format \emph{dictée-recopiage} est donc parfaitement logique compte tenu des contraintes techniques de l'époque. 

Inutile de dire que l'arrivée de l'imprimerie en Europe en 1455\cite{walsham2003} est un boulversement~:

\begin{coolquote}[Gutenberg, 1455\cite{quote-guten}]
Donnons des ailes à la vérité, qu'elle ne soit plus manuscrite à grands frais par des mains qui se fatiguent, mais qu'ils volent multipliés par une machine infatigable et qu'ils atteignent tous les hommes.
\end{coolquote}

L'opinion populaire se veut témoin de la colère de moines copistes ainsi dévalorisés, mais la vérité est moins simple. La possibilité de standardiser les textes intéresse beaucoup l'Église qui, en s'étirant sur une zone toujours plus grande, se voit menacé par les dérives idéalogiques locaux.

Il reste cependant quelques sceptiques parmi l'académique, notamment l'abbé Jean Trithème (1462-1516). En 1492 il rédige l'épitre \textbf{\emph{De laude scriptorum}} (éloge des scribes)~:

\begin{coolquote}[Jean Trithème\cite{monks-vs-press}\cite{in-praise-of-scribes}]
[Le scribe,] quand il transcrit un sujet religieux, est par l'acte même d'écrire initié en une certain mesure à la conaissance des grands mystères, et est grandement illuminé au plus profond de son âme; car les choses que nous écrivons sont plus fermement imprimés sur notre esprit\ldots quand il rumine par rapport aux Écritures il est fréquemment enflammée par eux.
\end{coolquote}

Or pour être véritablement \emph{enflammée} autour d'un sujet ne faut il pas arriver aux plus hautes sphères de la Taxonomie de Bloom, à l'\textbf{analyse} et à l'\textbf{évaluation}?

Il est à préciser que Trithème n'est pas contre l'imprimerie en soi. Cependant il ne pense pas qu'elle doit remplacer l'écriture manuelle, car ce travail est formateur\cite{abbot-trithemius}. Pour Trithème la valeur d'un ouvrage vient de l'effort émis pour le mettre en circulation, mais n'est-ce pas sur-valoriser le support au dépends du contenu?

Comme disait un certain célèbre faussaire artistique~: 

\begin{coolquote}[Van Meegeren, 1947\cite{magnusson2006fakers}]
Hier ce dessin valait des millions de florins, et experts et amateurs d'arts viendraient de partout dans le monde et paieraient pour le voir. Aujourd'hui, il ne vaut plus rien, et personne ne traverserait même la rue pour le voir gratuitement. Pourtant le dessin n'a pas changé. Alors pourquoi?
\end{coolquote}

Aujourd'hui la copie ne coûte plus rien, donc nous sommes ammenés à poser des questions vis-à-vis de la valeur d'un objet culturel. Créer artificiellement une rareté est-ce une pratique justifiable? Ne faudrait-il pas se limiter à la vente de la première copie? 

Nous devions également nous demander si la prise de note a de valeur maintenant que nous n'avons plus besoin, stricto-senso, de retranscire nos réferences à la main. Le recopiage a t'il une valeur mnémonique? Sert-il à comprendre en plus de mémoriser? Le format cours magistral est-il suranné aujourd'hui?  

\chapter{L'horloge maritime~: calcul automatique}
Nous pourrions remplir un rapport tout entier s'il était question d'énumérer les apports de l'outil logico-mathématique, donc nous nous limiterons à l'exemple pertinente des tables de logarithmes. Les logarithmes de Napier, introduits en 1614 dans son oeuvre \textbf{\emph{Mirifici Logarithmorum Canonis Descriptio}}, furent un outil abstrait visant à simplifier les calculs complexes. Les tables de logarithmes sont très vite devenus indispensables pour tout mathématicien, ingénieur, navigateur ou scientifique.

\begin{coolquote}[Pierre-Simon Laplace, 1614\cite{history-of-astronomy}]
[Les logarithmes sont] un artifice admirable qui, en réduisant à quelques jours un travail de plusieurs mois, double la durée de vie de l'astronome, et lui épargne les erreurs et le dégoût : plaies inséparables des longs calculs.
\end{coolquote}

\begin{minipage}[H]{0.49\linewidth}
  \begin{figure}[H]
  \centering
  \includegraphics[height=0.15\paperheight]{../resources/illustrations/napier}
  \caption{John Napier}
  \end{figure}
\end{minipage}
\begin{minipage}[H]{0.49\linewidth}
  \begin{figure}[H]
  \centering
  \includegraphics[height=0.15\paperheight]{../resources/illustrations/laplace}
  \caption{Pierre-Simon Laplace}
  \end{figure}
\end{minipage}
L'utilisation des logarithmes se repose sur des tables de précalcule tels l'\oe{}uvre de 1617 d'Henri Briggs. Muni d'une telle table la multiplication de deux nombres, par exemple, devient triviale~:

\begin{eqnarray}
                    &x            &= 2.16\times{8.13}              \nonumber \\
        \implies{}  &\log{x}      &= \log{2.16}+\log{8.13}         \nonumber \\
                    &             &= 0.3344548 + 0.9100905         \nonumber \\
                    &             &= 1.2445443                     \nonumber \\
        \implies{}  &x            &\approx{17.56}                   \nonumber \\
\end{eqnarray}

La table nous donne directement $\log{2.16}$ et $\log{8.13}$, et nous pouvoir également lire que le logarithme le plus proche est $log{17.56} = 1.2445245$. Nous en déduisons donc en quelques instants $2.16\times{8.13} \approx{17.56}$, ce qui n'est pas loin de de la vraie valeur $2.16\times{8.13}={17.5608}$. Il s'agit en somme d'un algorithme approximatif à base de tables de pré-calcule~: notons que de tels algorithmes existent depuis les Babyloniens\cite{robson-math}.

Laplace parle de l'astronome, car il s'agit de l'époque des grandes découvertes~: pour naviger on ne peut s'en passer des almanachs astrales et des algorithmes de navigation astronomique qui se reposaient dessus. Cependant si la latitude peut se calculer grace à l'hauteur perçu du soleil, la longitude est déterminé à partir de la conaissance de l'heure exacte en un endroit précis~: $4$ minutes de décalage correspond à une différence de longitude de $1^{\circ}$.  

\begin{minipage}[H]{0.59\linewidth}
  \begin{figure}[H]
  \centering
  \includegraphics[width=0.8\textwidth]{../resources/illustrations/scilly1707}
  \caption{Désastre naval de Sorlingues}
  \end{figure}
\end{minipage}
\begin{minipage}[H]{0.39\linewidth}
En 1714, après la perte en 1707 de 1,400 hommes et 4 navires à Sorlingues suite à un mauvais calcul de position, le parlement Britannique offre un prix de \pounds{20,000} à celui qui saura déterminer, avec une erreur maximum de 56 km, la longitude d'un navire en mer\cite{longitude}. 
%\vspace{1cm}
\end{minipage}

Entre en jeu le charpentier John Harrison, horlogier autodidacte qui construit 1736 la première horloge capable de fonctionner en voyage maritime, suivi d'autres toujours plus compactes et toujours plus précises. Cette invention fut cependant rejeté par l'orthodoxie académique de l'époque, qui voulaient impérativement une solution algorithmique-astrologique tel la méthode des distances lunaires, introduite en Angleterre en 1674 et perfectionnée par Nevil Maskelyne en 1767\cite{history-longitude}.

\begin{minipage}[H]{0.49\linewidth}
  \begin{figure}[H]
  \centering
  \includegraphics[height=0.15\paperheight]{../resources/illustrations/jharrison}
  \caption{John Harrison}
  \end{figure}
\end{minipage}
\begin{minipage}[H]{0.49\linewidth}
  \begin{figure}[H]
  \centering
  \includegraphics[height=0.15\paperheight]{../resources/illustrations/maskelyne}
  \caption{Nevil Maskelyne}
  \end{figure}
\end{minipage}

Harrison aura besoin d'attendre 1773 pour recevoir une prime réduite de \pounds{8,750}, et ne sera pas officiellement reconnu comme gagnant.

Pourqui ce rejet? Il s'agit d'un Homme ayant conçu un \og{}oracle\fg{} capable de résoudre le problème pour lui et non pas, distinction importante, une méthode lui permettant de le résoudre lui-même. L'informaticien devient alors prêtre d'un Dieu-machine plutôt que mathématicien-philosophe~: proposition controverse. Ce débat, entre ingénierie et science, trouve son écho aujourd'hui autour de l'apprentissage automatique~: nous pouvons concevoir des machines capable de reconaître des visages, sans comprendre pour autant comment fonctionne cette reconaissance. 

Notons également que de nos jours les algorithmes de calcul à base de tables logarithmiques ne sont même pas introduites à l'école~: la popularisation de calculatrices électroniques aux année 1970 leur ont rendus désuets, de même que l'écriture rend inutile l'épopée. Si des machines à calculer tels le Boulier existent depuis environ 4000 ans, ceux-ci ne sont véritablement que des supports mémoires~: c'est l'étudiant qui applique l'algorithme permettant de calculer le résultat désiré. 

Nous pouvions nous dire que le calcul mentale n'a peu d'importance maintenant que les machines à calcul sont omniprésent, mais la foi en une machine n'est pas moins dangereux que la foi en générale, s'il est sans mesure. 


\chapter{L'époque moderne et conclusion}

Au fil du temps les technologies de l'information n'ont eu de cesse de repousser les limites de
la propagation de l'information. 

Avec le téléphone, l'Homme peut désormais échanger \emph{en temps réelle} une 
information avec une autre personne \og{}~n'importe où~\fg{} dans le monde.

La télévision permet non seulement de faire de même avec des images, mais elle,
ou plutôt son usage, permet bien plus. Elle permet de préformater un document pour ensuite le
diffuser et éventuellement le rediffuser à l'attention d'un public nombreux et
hétérogène tel que l'on peut le faire avec un livre.

Puis vint Internet \og{}~International Network~\fg{}, en vrac: n'importe qui
peut diffuser un document pour un coup très faible, une quantité astronomique
de documents accessibles, Internet se présente comme un méta-outil et offre de
nombreux outils de communication (Mail, Forum, chat, VoIP, conférence, etc...)



