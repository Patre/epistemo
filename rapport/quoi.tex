% !TEX encoding = UTF-8 Unicode
% !TEX root = rapport.tex

\part{Historique}\label{quoi}

\section*{Introduction : protocoles de communication}
Avec l'apparition du langage, l'être humain s'est doté d'un outil capable d'exprimer des informations de plus en plus complexes. Environ 6000 langues sont parlées aujourd'hui ; les anthropologues n'ont jamais découvert un peuple démuni d'un langage complexe\cite{linguistics-pinker}. 

\begin{minipage}[H]{0.49\linewidth}
  \begin{figure}[H]
  \centering
  \includegraphics[width=0.8\textwidth]{../resources/illustrations/psamtik-I}
  \caption{Pharaon Psammétique~I}
  \end{figure}
\end{minipage}
\begin{minipage}{0.5\linewidth}
Il est difficile de dater l'apparition du langage. L'Homme essaye pourtant depuis des milliers d'années de trouver un \emph{proto-language} qui serait à l'origine de tous les autres, mais sans succès. 
\vspace{1cm}
\end{minipage}

\begin{minipage}{0.5\linewidth}
Dans \og L'Enquête\fg{}, l'historien Grec Hérodote fait référence à une expérience de \emph{privation de langage} du pharaon Psammétique~I visant à identifier une sorte de langage \emph{inné} de l'Homme\cite{herodote-privation}. 
\vspace{1cm}
\end{minipage}
\begin{minipage}[H]{0.49\linewidth}
  \begin{figure}[H]
  \centering
  \includegraphics[width=0.8\textwidth]{../resources/illustrations/herodote}
  \caption{Historien Hérodote}
  \end{figure}
\end{minipage}

\begin{minipage}[H]{0.49\linewidth}
  \begin{figure}[H]
  \centering
  \includegraphics[width=0.8\textwidth]{../resources/illustrations/fred-II}
  \caption{Empereur Frédéric~II}
  \end{figure}
\end{minipage}
\begin{minipage}[H]{0.5\linewidth}
De tels expériences furent répétées au fil de l'histoire, notamment par l'empereur Frédéric~II de Hohenstaufen, pour qui le résultat fût la mort des nourrissons volontaires\cite{ggcoulton-francis-to-dante}. 
\end{minipage}

\begin{minipage}[H]{0.65\linewidth}
Des exemples plus modernes d'\emph{enfants sauvages} ont conduit à l'hypothèse de la \emph{période critique} de Wilder Penfield et Lamar Roberts\cite{penfield2003speech}, popularisée par Eric Lenneberg\cite{lenneberg-crit-period}.

Selon cette théorie, un enfant privé de vocalisation pendant les premières années de sa vie sera incapable de bien assimiler le langage par la suite.

\end{minipage}
\begin{minipage}[H]{0.34\linewidth}
  \begin{figure}[H]
  \centering
  \includegraphics[width=0.8\textwidth]{../resources/illustrations/penfield}
  \caption{Wilder Penfield}
  \end{figure}
\end{minipage}

Peu importe sa ou ses origines, le langage est une technologie de l'information primordiale. Il permet un transfert d'informations entre membres d'une société, et donc, par le bias des traditions orales, un stockage de l'information pendant une durée supérieure à la vie d'un individu. Ainsi munis de la capacité de se propager en dehors de leurs hôtes, les mèmes deviennent alors potentiellement immortels.

Chez les peuples illettrés, une deuxième invention est souvent utilisée pour faciliter partage et mémorisation des récits sous forme de poésies rythmées : la structure d'épopée\cite{havelock-preface-plato}. Notons que la poésie est toujours utilisée aujourd'hui comme exercice de mémorisation (à l'école) et moyen mnémotechnique (publicité par exemple).

\chapter{L'écriture~: mémoire externe}
Si le langage peut être vu comme un instinct plutôt qu'une technologie, ce n'est pas le cas de l'écriture. En effet, l'écriture idéographique est apparu il y a environ 5000 ans chez un nombre limité de civilisations. L'écriture \emph{alphabétique} semble d'ailleurs n'être apparu qu'une seule fois : chez les Cananéens il y a 3700 ans\cite{linguistics-pinker}. Toute autre écriture alphabétique serait donc dérivée de celle-ci.
%\gls{Écriture alphabétique~: écriture où chaque symbole correspond à un son vocal.}

Dans tous les cas l'invention permet à un support mort de stoker un ensemble de données codées sous forme de symboles, et donc de repousser d'avantage la frontière de l'espace et du temps. Cette avancé rend presque redondantes les techniques de mémorisation lyriques mentionnées ci-dessus. Certaines réactions de ceux qui s'y sont investis sont donc peu surprenantes~:

\begin{coolquote}[Socrates (attribué)\cite{plato-phaedrus}]
Elle ne peut produire dans les âmes, en effet, que l'oubli de ce qu'elles  savent en leur faisant négliger la mémoire. Parce qu'ils auront foi dans  l'écriture, c'est par le dehors, par des empreintes étrangères, et non plus du dedans et du fond d'eux-mêmes, que les hommes chercheront à se ressouvenir. 
\end{coolquote}

\begin{minipage}[H]{0.3\linewidth}
  \begin{figure}[H]
  \centering
  \includegraphics[width=0.8\textwidth]{../resources/illustrations/bloom_face}
  \caption{Benjamin Bloom}
  \end{figure}
\end{minipage}
\begin{minipage}[H]{0.69\linewidth}
Cette critique, faite à travers un dialogue entre Socrates et Phaedrus, suggère que l'écriture, en plus de nuire à la mémoire, limite son lecteur à ce que la taxonomie de Bloom\cite{tax-bloom} appelerait \emph{connaître} à la différence de \emph{comprendre}, \emph{appliquer}, etc. 
\vspace{1cm}
\end{minipage}

\begin{coolquote}[Socrates (attribué), -370\cite{plato-phaedrus}]
Tu as trouvé le moyen, non point d'enrichir la mémoire, mais de conserver les souvenirs qu'elle a. Tu donnes à tes disciples la présomption qu'ils ont la science, non la science elle-même. Quand ils auront, en effet, beaucoup appris sans maître, ils s'imagineront devenus très savants, et ils ne seront pour la plupart que des ignorants de commerce incommode, des savants imaginaires au lieu de vrais savants.
\end{coolquote}

\begin{minipage}[H]{0.49\linewidth}
  \begin{figure}[H]
  \centering
  \includegraphics[height=0.15\paperheight]{../resources/illustrations/plato2}
  \caption{Plato}
  \end{figure}
\end{minipage}
\begin{minipage}[H]{0.49\linewidth}
  \begin{figure}[H]
  \centering
  \includegraphics[height=0.15\paperheight]{../resources/illustrations/socrates}
  \caption{Socrates}
  \end{figure}
\end{minipage}

Pour Platon une connaissance ne peut être transférée correctement sans être assimilée. Or le papier, si bien soit-il capable de stoker des propos, est incapable de les comprendre ou de les défendre.

Cependant, il ne faut pas oublier que Platon était un professeur dont la pédagogie reposait sur le dialogue et non la lecture. Il n'est donc pas un interlocuteur très objective. Notons également avec ironie que si nous connaissons Platon c'est grâce à l'écriture, et si nous confondons Socrates avec lui c'est que ce dernier n'a laissé aucune trace écrite.

Cela étant, les critiques de Platon restent pertinentes aujourd'hui. Elles le deviennent d'autant plus étant donné que la consultation d'informations est maintenant tellement facile que la mémorisation semble obsolète. Mais il faut bien distinguer la différence entre effleurer un propos et l'assimiler.

Quand nous nous approprions véritablement d'un propos il devient bien plus qu'une copie supplémentaire redondante~: l'idée nous appartient, nous change et est changée pour nous\ldots

\chapter{L'imprimerie~: reproduction des données}

Malgré les réticences des philosophes Grecs, l'éducation en Europe pendant le Moyens Âge repose sur l'écrit et non le discours, l'académie de l'époque se voyant comme conservateur de patrimoine classique et non chercheur de nouvelles idées\cite{friesen-the-lecture}.

L'enseignement se fait alors par le biais de \emph{lectures}, mot qui est d'ailleurs toujours utilisé en Anglais pour dire \emph{cours magistral}\textbf{~: c'est d'ailleurs à ce moment que cette forme de pédagogie apparait, surtout pour faire face à la rareté de textes}.

En effet à l'époque le livre est un véritable objet d'art, un trésor du patrimoine fait pour survivre durant plusieurs générations de bibliothécaires~: chaque ouvrage est retranscrit et décoré avec soin par un copiste artisan. Le futur savant ne pouvant pas s'offrir tous les textes dont il aura besoin par la suite, il doit les retranscrire lui-même\cite{friesen-the-lecture}. Le format \emph{dictée-recopiage} est donc pertinent compte tenu des contraintes techniques de l'époque. 

Inutile de dire que l'arrivée de l'imprimerie en Europe en 1455\cite{walsham2003} est un bouleversement~:

\begin{coolquote}[Johannes Gutenberg, 1455\cite{quote-guten}]
Donnons des ailes à la vérité, qu'elle ne soit plus manuscrite à grands frais par des mains qui se fatiguent, mais qu'ils volent multipliés par une machine infatigable et qu'ils atteignent tous les hommes.
\end{coolquote}

\begin{minipage}[H]{0.39\linewidth}
L'opinion populaire se veut témoin de la colère de moines copistes ainsi dévalorisés, mais la vérité est moins simple. 

La possibilité de standardiser les textes intéresse beaucoup l'Église qui, en s'étirant sur une zone toujours plus grande, se voit menacée par les dérives idéologiques locaux.
\vspace{1cm}
\end{minipage}
\begin{minipage}[H]{0.59\linewidth}
  \begin{figure}[H]
  \centering
  \includegraphics[height=0.15\paperheight]{../resources/illustrations/gutenberg}
  \caption{Johannes Gutenberg}
  \end{figure}
\end{minipage}

Il reste cependant quelques sceptiques parmi l'académie, notamment l'abbé Jean Trithème (1462--1516) connu pour sa stéganographie. En 1492 il rédige l'épitre \og De laude scriptorum \fg{} (éloge des scribes)~:

\begin{coolquote}[Jean Trithème\cite{monks-vs-press}\cite{in-praise-of-scribes}]
[Le scribe,] quand il transcrit un sujet religieux, est par l'acte même d'écrire initié en une certain mesure à la connaissance des grands mystères, et est grandement illuminé au plus profond de son âme; car les choses que nous écrivons sont plus fermement imprimés sur notre esprit\ldots quand il rumine par rapport aux Écritures il est fréquemment enflammée par eux.
\end{coolquote}

\begin{minipage}[H]{0.54\linewidth}
Or pour être \textbf{véritablement \emph{enflammée} autour }d'un sujet ne faut il pas arriver aux plus hautes sphères de la Taxonomie de Bloom, à l'\textbf{analyse} et à l'\textbf{évaluation} ?

Il faut préciser que Trithème n'était pas contre l'imprimerie en soi. Cependant il pense qu'elle ne devrait pas remplacer l'écriture manuelle, ce travail étant formateur\cite{abbot-trithemius}. Pour Trithème la valeur d'un ouvrage vient de l'effort émit pour le mettre en circulation. Mais n'est-ce pas sur-valoriser le support au dépends du contenu ?
\vspace{1cm}
\end{minipage}
\begin{minipage}[H]{0.44\linewidth}
  \begin{figure}[H]
  \centering
  \includegraphics[height=0.15\paperheight]{../resources/illustrations/trithemius}
  \caption{Jean Trithème}
  \end{figure}
\end{minipage}

Comme disait un certain célèbre faussaire artistique~: 

\begin{coolquote}[Van Meegeren, 1947\cite{magnusson2006fakers}]
Hier ce dessin valait des millions de florins, et experts et amateurs d'arts viendraient de partout dans le monde et paieraient pour le voir. Aujourd'hui, il ne vaut plus rien, et personne ne traverserait même la rue pour le voir gratuitement. Pourtant le dessin n'a pas changé. Alors pourquoi?
\end{coolquote}

\begin{minipage}[H]{0.44\linewidth}
Aujourd'hui la copie ne coûte plus rien, donc nous sommes amenés à nous poser des questions vis-à-vis de la valeur d'un objet culturel. Créer artificiellement une rareté est-ce une pratique justifiable ? Ne faudrait-il pas se limiter à la vente de la première copie ?
\vspace{1cm}
\end{minipage}
\begin{minipage}[H]{0.54\linewidth}
  \begin{figure}[H]
  \centering
  \includegraphics[height=0.15\paperheight]{../resources/illustrations/meergeren}
  \caption{Han Van Meergeren}
  \end{figure}
\end{minipage}

Nous devons également nous demander si la prise de note a toujours une valeur alors que nous n'avons plus besoin, stricto-senso, de retranscrire nos références à la main. Le recopiage a t-il une valeur mnémonique ? Favorise t-il la compréhension en plus de la mémorisation ? Le format du cours magistral est-il dépassé aujourd'hui ?

\chapter{L'horloge maritime~: calcul automatique}
Nous pourrions remplir un rapport tout entier s'il était question d'énumérer les apports de l'outil logico-mathématique. Nous nous limiterons donc à l'exemple pertinent des tables de logarithmes. Les logarithmes de Napier, introduits en 1614 dans son œuvre \og  Mirifici Logarithmorum Canonis Descriptio \fg{}, furent un outil abstrait visant à simplifier les calculs complexes. Les tables de logarithmes sont vite devenues indispensables pour tout mathématicien, ingénieur, navigateur ou scientifique.

\begin{coolquote}[Pierre-Simon Laplace, 1614\cite{history-of-astronomy}]
[Les logarithmes sont] un artifice admirable qui, en réduisant à quelques jours un travail de plusieurs mois, double la durée de vie de l'astronome, et lui épargne les erreurs et le dégoût : plaies inséparables des longs calculs.
\end{coolquote}

\begin{minipage}[H]{0.49\linewidth}
  \begin{figure}[H]
  \centering
  \includegraphics[height=0.15\paperheight]{../resources/illustrations/napier}
  \caption{John Napier}
  \end{figure}
\end{minipage}
\begin{minipage}[H]{0.49\linewidth}
  \begin{figure}[H]
  \centering
  \includegraphics[height=0.15\paperheight]{../resources/illustrations/laplace}
  \caption{Pierre-Simon Laplace}
  \end{figure}
\end{minipage}
L'utilisation des logarithmes se repose sur des tables de pré-calcule comme l'\oe{}uvre de 1617 d'Henri Briggs. Muni d'une telle table la multiplication de deux nombres, par exemple, devient triviale~:

\begin{eqnarray}
                    &x            &= 2.16\times{8.13}              \nonumber \\
        \implies{}  &\log{x}      &= \log{2.16}+\log{8.13}         \nonumber \\
                    &             &= 0.3344548 + 0.9100905         \nonumber \\
                    &             &= 1.2445443                     \nonumber \\
        \implies{}  &x            &\approx{17.56}                   \nonumber \\
\end{eqnarray}

La table nous donne directement $\log{2.16}$ et $\log{8.13}$, et nous pouvons également lire que le logarithme le plus proche est $log{17.56} = 1.2445245$. Nous en déduisons donc en quelques instants $2.16\times{8.13} \approx{17.56}$, une approximation proche de la vraie valeur $2.16\times{8.13}={17.5608}$. Il s'agit en fait d'un algorithme approximatif à base de tables de pré-calcule. Notons que de tels algorithmes existent depuis les Babyloniens\cite{robson-math}.

Laplace parle de l'astronome, car il s'agit de l'époque des grandes découvertes~: pour naviguer on ne peut se passer des almanachs astrales et des algorithmes de navigation astronomique qui reposaient dessus. Cependant si la latitude peut se calculer grâce à la hauteur perçue du soleil, la longitude est déterminée à partir de la connaissance de l'heure exacte en un endroit précis~: $4$ minutes de décalage correspond à une différence de longitude de $1^{\circ}$.  

\begin{minipage}[H]{0.59\linewidth}
  \begin{figure}[H]
  \centering
  \includegraphics[width=0.8\textwidth]{../resources/illustrations/scilly1707}
  \caption{Désastre naval de Sorlingues}
  \end{figure}
\end{minipage}
\begin{minipage}[H]{0.39\linewidth}
En 1714, après la perte de 1,400 hommes et 4 navires sept ans plus tôt à Sorlingues suite à un mauvais calcul de position, le parlement Britannique a offert un prix de \pounds{20,000} à celui qui saura déterminer, avec une erreur maximum de 56 km, la longitude d'un navire en mer\cite{longitude}. 
%\vspace{1cm}
\end{minipage}

Entre en jeu le charpentier John Harrison, horloger autodidacte. Il construit en 1736 la première horloge capable de fonctionner en voyage maritime, suivie d'autres toujours plus compactes et toujours plus précises. Cette invention fût cependant rejetée par l'orthodoxie académique de l'époque, qui voulaient impérativement une solution algorithmique-astrologique comme la méthode des distances lunaires, introduite en Angleterre en 1674 et perfectionnée par Nevil Maskelyne en 1767\cite{history-longitude}.

\begin{minipage}[H]{0.49\linewidth}
  \begin{figure}[H]
  \centering
  \includegraphics[height=0.15\paperheight]{../resources/illustrations/jharrison}
  \caption{John Harrison}
  \end{figure}
\end{minipage}
\begin{minipage}[H]{0.49\linewidth}
  \begin{figure}[H]
  \centering
  \includegraphics[height=0.15\paperheight]{../resources/illustrations/maskelyne}
  \caption{Nevil Maskelyne}
  \end{figure}
\end{minipage}

Harrison aura besoin d'attendre 1773 pour recevoir une prime réduite de \pounds{8,750}, et ne sera pas officiellement reconnu gagnant.

Pourquoi ce rejet? Il s'agit d'un Homme ayant conçu un \og oracle\fg{} capable de résoudre le problème pour lui et non pas, distinction importante, une méthode lui permettant de le résoudre lui-même. L'informaticien devient alors prêtre d'un Dieu-machine plutôt que mathématicien-philosophe~: proposition controverse. Ce débat, entre ingénierie et science, trouve son écho aujourd'hui autour de l'apprentissage automatique~: nous pouvons concevoir des machines capable de reconnaître des visages, sans comprendre pour autant comment fonctionne cette reconnaissance. 

Notons également que de nos jours les algorithmes de calcul à base de tables logarithmiques ne sont pas introduites à l'école~: la popularisation de calculatrices électroniques dans les années 1970 les ont rendues désuets, de même que l'écriture a rendue inutile l'épopée. Si des machines à calculer tel que le boulier existent depuis environ 4000 ans, ceux-ci ne sont véritablement que des supports mémoires~: c'est l'étudiant qui applique l'algorithme permettant de calculer le résultat désiré. 

Nous pouvons nous demander si le calcul mentale a encore de l'importance maintenant que les machines à calcul sont omniprésentes, mais la foi en une machine n'est pas moins dangereux que la foi en générale, s'il est sans mesure\ldots


\chapter*{Conclusion~: époque moderne}

Au fil du temps les technologies de l'information n'ont eu de cesse de repousser les limites de la propagation de l'information. Nous pourrions citer le téléphone, la radio, la télévision, l'internet, etc. Mais finalement nous ne trouvons en ces technologies pas de problématique pédagogique qui n'ai pas déjà été révélée dans le passée.

Ce que nous appelons aujourd'hui l'\emph{outil informatique} n'est qu'un assemblage perfectionné de l'encodage, du copiage et du calcul, le tout automatisé, facile d'utilisation et disponible à tous (au moins dans les pays développés). Les mêmes questions se posent donc à nouveau~:

\begin{itemize}
\Large
\item Faut-il mémoriser?

\item Faut-il recopier?

\item Faut-il calculer?

\item \textbf{Faut-il changer l'éducation?}

\item \textbf{Si oui, comment?}

\end{itemize}

Dans les deux parties qui suivent nous nous tâcherons de répondre respectivement à ces deux dernières questions.