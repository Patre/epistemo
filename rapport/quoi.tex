% !TEX encoding = UTF-8 Unicode
% !TEX root = rapport.tex

\chapter{Historique de l'intégration des technologies de l'information \ldots}\label{quoi}

\section{Avant-propos~: Technologies \og{}Informatiques\fg{}?}

Le mot \og{}informatique\fg{} est une concatenation d'\og{}information\fg{} et \og{}automatique\fg{} fait en 1957 par Karl Steinbuch\cite{steinbuch-2005} pour décrire le traitement automatique de l'information. 
Depuis le terme a été adopté pour décrire une gamme tellement vaste de sciences, de technologies et de services qu'il a besoin d'être qualifié pour avoir un sens précis. D'après Wikipedia\cite{wiki-informatique}~:
\begin{quote}
Les expressions \og{}science informatique\fg{}, \og{}informatique fondamentale\fg{} ou \og{}informatique théorique\fg{} sont utilisées pour désigner sans ambiguïté la science, tandis que \og{}technologies de l'information\fg{} ou \og{}technologies de l'information et de la communication\fg{} désignent le secteur industriel et ses produits.
\end{quote}
Ici nous utiliserons \og{}technologies de l'information\fg{} pour décrire toute technique permettant de stoquer, de traiter ou de transférer l'information. Cette définition couvre évidemment les algorithmes, structure données et protocoles de communication, mais aussi les langages naturels, l'écriture, et tout outil de réflexion logico-mathématique\ldots

\section{La naissance du langage}
L'Homme a essaié pendant pendant des milliers d'années de trouver un \og{}proto-language\fg{}, origine de toute autre, et de dater son apparition. Dans \og{}L'Enquête\fg{}, l'historien Grecque Hérodote fait référence à une expérience de \og{}privation de langage\fg{} du pharaon Psammétique I visant à déterminer le langage \og{}par défaut\fg{} de l'Homme\cite{herodote-privation}. De tels expériences furènt répétés au fil de l'histoire, notamment par l'empereur Frédéric II de Hohenstaufen, pour qui le résultat fut la mort de ses cobayes\cite{ggcoulton-francis-to-dante}. Les exemples plus modernes d'enfants \og{}sauvages\fg{} ont conduites à l'hypothèse de la \og{}période critique\fg{} de Eric Lenneberg\cite{lenneberg-crit-period}~: un enfant privé de vocalisations pendant les première années de sa vie sera incapable de bien assimiler le langage par la suite.

Peu importe son ou ses origines, le langage est une technologie de l'information primordiale. Il permet un transfert d'informations entre membres d'une société mais aussi, par le bias des traditions orales, un stockage de l'information pendant une durée théoriquement infini. Chez les peuples alettrés une deuxième technologie, la structure d'épopée, est souvent utilisé pour faciliter partage et mémorisation des récits sous form de poésies rythmés \cite{havelock-preface-plato}. Nous pouvions d'ailleurs noter que la poésie est toujours utilisé aujourd'hui comme exercise de mémorisation (à l'école) et moyen mnémotechnique (dans la publicité).

 

Apparu vers 3500 avant J.-C\cite{dreyer00}, l'écriture est [une des premières grandes](euh en
faite c'est la principale, tous repose sur elle) \glspl{technologie} de
l'information. 

Bien qu'aujourd'hui celle-ci nous parait totalement
indissociable de l'éducation, cette invention fut critiqué notamment par Platon
à travers un dialogue entre Socrates et Phaedrus. Pour lui, se reposer sur
l'écriture nuit à la mémoire. Ironiquement, si nous connaissons Socrates c'est
grâce aux écrits de Platon, son élève. Cf. aussi le 7ème Epitre de Platon.


%%@? Ne faudrait-il pas commencer la partie par cette citation ?
\begin{quote}
\og{}~Elle ne peut produire dans les âmes, en effet, que l’oubli de ce qu’elles 
savent en leur faisant négliger la mémoire. Parce qu’ils auront foi dans 
l’écriture, c’est par le dehors, par des empreintes étrangères, et non plus du 
dedans et du fond d’eux-mêmes, que les hommes chercheront à se ressouvenir. Tu 
as trouvé le moyen, non point d’enrichir la mémoire, mais de conserver les 
souvenirs qu’elle a. Tu donnes à tes disciples la présomption qu’ils ont la 
science, non la science elle-même. Quand ils auront, en effet, beaucoup appris 
sans maître, ils s’imagineront devenus très savants, et ils ne seront pour la 
plupart que des ignorants de commerce incommode, des savants imaginaires au lieu
 de vrais savants.~\fg{}
\url{https://cercamon.wordpress.com/2006/05/10/platon-sur-lecriture-phedre-274-276/}
(Plato, Phaedrus 275a-b)
\end{quote}

\subsection{Les tournant majeur des technologies de l'information}
//@? Comment introduire la notion de document ?

langage - écriture - imprimante - téléphone - télévision - internet


Les technologies de l'information n'ont eu de cesse de repousser les limites de
la propagation de l'information, avec l'apparition du langage l'être humain
s'est doté d'un outil capable d'exprimer des informations de plus en plus
complexe. 

Dans un second temps l'apparition de l'écriture à permis de repoussé les
frontière de l'espace et du temps… (dans une certaine mesure)

Avec l'arrivée de l'imprimante, l'Homme est désormais capable de copier
rapidement une information pour la \emph{diffuser}. (Apparition d'une sorte
de broadcast)

Avec le téléphone, l'Homme peut désormais échanger \emph{en tant réelle} une 
information avec une autre personne \og{}~n'importe où~\fg{} dans le monde.

La télévision permet non seulement de faire de même avec des images, mais elle,
ou plutôt son usage, permet bien plus. Elle permet de préformater un document pour ensuite le
diffuser et éventuellement le rediffuser à l'attention d'un public nombreux et
hétérogène tel que l'on peut le faire avec un livre.

Puis vint Internet \og{}~International Network~\fg{}, en vrac: n'importe qui
peut diffuser un document pour un coup très faible, une quantité astronomique
de documents accessibles, Internet se présente comme un méta-outil et offre de
nombreux outils de communication (Mail, Forum, chat, VoIP, conférence, etc...)

\subsection{en vrac}

Les logarithmes de Napier, introduits en 1614 dans son oeuvre
"Mirifici Logarithmorum Canonis Descriptio", furent un outil de calcul
abstrait visant à simplifier des calculs complexes. À l'époque les
tables de logarithmes (Henry Briggs 1617) apprises par coeur [citation
  needed] étaient indispensables pour tout mathématicien, ingénieur ou
scientifique. De nos jours ces tables ne sont mêmes plus introduites à
l'école.

Des débats intenses suivent la popularisation de la calculatrice électronique 
aux années 1970: suit un débat. Il est important de se rappeler que les 
machines à calculer tels l'Abacus existent depuis environ 4000 ans. Ceux-ci 
portent cependant certaines avantages pédagogiques [citation needed]: c'est 
l'usager qui déroule l'algorithme et non la machine qui le fait toute seule. 

Depuis les années 90: internet, outil pour tricher?
\url{http://www.apsq.org/sautquantique/doss/d-tricherie.html/}

\ldots

\section{\ldots dans la société}

La notion de nation est-elle pertinente dans un monde connecté par
internet~?

"The Big Switch: Rewiring the World from Edison to Google" 
 \url{http://www.nicholasgcarr.com/bigswitch/}

Nous perdons notre confidentialité de l'intime. Pour Google, Facebook, etc, 
ce n'est pas un problème: ceux qui n'ont rien à cacher n'ont pas à se soucier. Nous
pouvons cependant noter qu'ils ont intérêt à penser ainsi [j'arrive pas à 
trouver l'article qui en parle]. Le voyeurisme de la télé-réalité montre 
que ce phénomène est de Zeitgeist (esprit de l'époque): nous nous soucions 
moins, par exemple, de qui nous entend parler lors de nos
conversations depuis un téléphone portable.

Les réseaux sociaux permettent un exhibitionnisme, et ceci conduit au 
narcissisme:
http://www.guardian.co.uk/technology/2012/mar/17/facebook-dark-side-study-aggressive-narcissism 
[WHY U NO GIVE ME LINK TO ORIGINAL ARTICLE!]

"The Spy in the Coffee Machine: The End of Privacy As We Know It"
 \url{http://eprints.soton.ac.uk/265683/}

En Février 2010 Eben Moglen lance la notion de "Freedom Box" pour faire face à
la centralisation progressive du web. C'est d'ailleurs sa présentation qui fut
l'amorce du projet Diaspora, visant à créer une plateforme sociale décentralisé 
respectant la confidentialité de ses usagers. 

\subsection{\ldots par les individus}


