% !TEX encoding = UTF-8 Unicode
% !TEX root = rapport.tex

\chapter{Historique de l'intégration des technologies de l'information \ldots}\label{quoi}

\section{\ldots dans l'éducation}

Apparu vers 3500 avant J.-C\cite{dreyer00}, l'écriture est [une des premières grandes](euh en
faite c'est la principale, tous repose sur elle) \glspl{technologie} de
l'information. Bien qu'aujourd'hui celle-ci nous paraissent totalement
indissociable de l'éducation, cette invention fut critiqué notamment par Platon
à travers un dialogue entre Socrates et Phaedrus. Pour lui, se reposer sur
l'écriture nuit à la mémoire. Ironiquement, si nous connaissons Platon c'est
grâce à ses écrits. Cf. aussi le 7ème Epitre de Platon.


%%@? Ne faudrait-il pas commencer la partie par cette citation ?
\begin{quote}
\og{}~Elle ne peut produire dans les âmes, en effet, que l’oubli de ce qu’elles 
savent en leur faisant négliger la mémoire. Parce qu’ils auront foi dans 
l’écriture, c’est par le dehors, par des empreintes étrangères, et non plus du 
dedans et du fond d’eux-mêmes, que les hommes chercheront à se ressouvenir. Tu 
as trouvé le moyen, non point d’enrichir la mémoire, mais de conserver les 
souvenirs qu’elle a. Tu donnes à tes disciples la présomption qu’ils ont la 
science, non la science elle-même. Quand ils auront, en effet, beaucoup appris 
sans maître, ils s’imagineront devenus très savants, et ils ne seront pour la 
plupart que des ignorants de commerce incommode, des savants imaginaires au lieu
 de vrais savants.~\fg{}
\url{https://cercamon.wordpress.com/2006/05/10/platon-sur-lecriture-phedre-274-276/}
(Plato, Phaedrus 275a-b)
\end{quote}

\subsection{Les tournant majeur des technologies de l'information}
//@? Comment introduire la notion de document ?

langage - écriture - imprimante - téléphone - télévision - internet


Les technologies de l'information n'ont eu de cesse de repousser les limites de
la propagation de l'information, avec l'apparition du langage l'être humain
s'est doté d'un outil capable d'exprimer des informations de plus en plus
complexe. 

Dans un second temps l'apparition de l'écriture à permis de repoussé les
frontière de l'espace et du temps… (dans une certaine mesure)

Avec l'arrivée de l'imprimante, l'Homme est désormais capable de copier
rapidement une information pour la \emph{diffuser}. (Apparition d'une sorte
de broadcast)

Avec le téléphone, l'Homme peut désormais échanger \emph{en tant réelle} une 
information avec une autre personne \og{}~n'importe où~\fg{} dans le monde.

La télévision permet non seulement de faire de même avec des images, mais elle,
ou plutôt son usage, permet bien plus. Elle permet de préformater un document pour ensuite le
diffuser et éventuellement le rediffuser à l'attention d'un public nombreux et
hétérogène tel que l'on peut le faire avec un livre.

Puis vint Internet \og{}~International Network~\fg{}, en vrac: n'importe qui
peut diffuser un document pour un coup très faible, une quantité astronomique
de documents accessibles, Internet se présente comme un méta-outil et offre de
nombreux outils de communication (Mail, Forum, chat, VoIP, conférence, etc...)

\subsection{en vrac}

Les logarithmes de Napier, introduits en 1614 dans son oeuvre
"Mirifici Logarithmorum Canonis Descriptio", furent un outil de calcul
abstrait visant à simplifier des calculs complexes. À l'époque les
tables de logarithmes (Henry Briggs 1617) apprises par coeur [citation
  needed] étaient indispensables pour tout mathématicien, ingénieur ou
scientifique. De nos jours ces tables ne sont mêmes plus introduites à
l'école.

Des débats intenses suivent la popularisation de la calculatrice électronique 
aux années 1970: suit un débat. Il est important de se rappeler que les 
machines à calculer tels l'Abacus existent depuis environ 4000 ans. Ceux-ci 
portent cependant certaines avantages pédagogiques [citation needed]: c'est 
l'usager qui déroule l'algorithme et non la machine qui le fait toute seule. 

Depuis les années 90: internet, outil pour tricher?
\url{http://www.apsq.org/sautquantique/doss/d-tricherie.html/}

\ldots

\section{\ldots dans la société}

La notion de nation est-elle pertinente dans un monde connecté par
internet~?

"The Big Switch: Rewiring the World from Edison to Google" 
 \url{http://www.nicholasgcarr.com/bigswitch/}

Nous perdons notre confidentialité de l'intime. Pour Google, Facebook, etc, 
ce n'est pas un problème: ceux qui n'ont rien à cacher n'ont pas à se soucier. Nous
pouvons cependant noter qu'ils ont intérêt à penser ainsi [j'arrive pas à 
trouver l'article qui en parle]. Le voyeurisme de la télé-réalité montre 
que ce phénomène est de Zeitgeist (esprit de l'époque): nous nous soucions 
moins, par exemple, de qui nous entend parler lors de nos
conversations depuis un téléphone portable.

Les réseaux sociaux permettent un exhibitionnisme, et ceci conduit au 
narcissisme:
http://www.guardian.co.uk/technology/2012/mar/17/facebook-dark-side-study-aggressive-narcissism 
[WHY U NO GIVE ME LINK TO ORIGINAL ARTICLE!]

"The Spy in the Coffee Machine: The End of Privacy As We Know It"
 \url{http://eprints.soton.ac.uk/265683/}

En Février 2010 Eben Moglen lance la notion de "Freedom Box" pour faire face à
la centralisation progressive du web. C'est d'ailleurs sa présentation qui fut
l'amorce du projet Diaspora, visant à créer une plateforme sociale décentralisé 
respectant la confidentialité de ses usagers. 

\subsection{\ldots par les individus}


