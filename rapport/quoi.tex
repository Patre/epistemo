% !TEX encoding = UTF-8 Unicode
% !TEX root = rapport.tex

\chapter{Historique de l'intégration des technologies de l'information \ldots}\label{quoi}

\section*{Avant-propos~: Technologies \og{}Informatiques\fg{}?}
Le mot \og{}informatique\fg{} est une concatenation d'\og{}information\fg{} et \og{}automatique\fg{} fait en 1957 par Karl Steinbuch\cite{steinbuch-2005} pour décrire le traitement automatique de l'information. 
Depuis le terme a été adopté pour décrire une gamme tellement vaste de sciences, de technologies et de services qu'il a besoin d'être qualifié pour avoir un sens précis. D'après Wikipedia\cite{wiki-informatique}~:
\begin{quote}
Les expressions \og{}science informatique\fg{}, \og{}informatique fondamentale\fg{} ou \og{}informatique théorique\fg{} sont utilisées pour désigner sans ambiguïté la science, tandis que \og{}technologies de l'information\fg{} ou \og{}technologies de l'information et de la communication\fg{} désignent le secteur industriel et ses produits.
\end{quote}
Ici nous utiliserons \og{}technologies de l'information\fg{} pour décrire toute technique permettant de stoquer, de traiter ou de transférer l'information. Cette définition couvre évidemment les algorithmes, structure données et protocoles de communication, mais aussi les langages naturels, l'écriture, et tout outil de réflexion logico-mathématique\ldots

\section{La naissance du langage}
Avec l'apparition du langage l'Être humain s'est doté d'un outil capable d'exprimer des informations de plus en plus complexes. Nous essayons sans succès depuis des milliers d'années de trouver un \og{}proto-language\fg{}, origine de toute autre, et de dater son apparition. Dans \og{}L'Enquête\fg{}, l'historien Grecque Hérodote fait référence à une expérience de \og{}privation de langage\fg{} du pharaon Psammétique I visant à déterminer le langage \og{}par défaut\fg{} de l'Homme\cite{herodote-privation}. De tels expériences furènt répétés au fil de l'histoire, notamment par l'empereur Frédéric II de Hohenstaufen, pour qui le résultat fut la mort de ses \og{}cobayes\fg{}\cite{ggcoulton-francis-to-dante}. Les exemples plus modernes d'enfants \og{}sauvages\fg{} ont conduits à l'hypothèse de la \og{}période critique\fg{} d'Eric Lenneberg\cite{lenneberg-crit-period}~: un enfant privé de vocalisations pendant les première années de sa vie sera incapable de bien assimiler le langage par la suite.

Peu importe son ou ses origines, le langage est une technologie de l'information primordiale. Il permet un transfert d'informations entre membres d'une société mais aussi, par le bias des traditions orales, un stockage de l'information pendant une durée théoriquement infini. Chez les peuples alettrés une deuxième invention, la structure d'épopée, est souvent utilisé pour faciliter partage et mémorisation des récits sous form de poésies rythmés \cite{havelock-preface-plato}. Notons d'ailleurs que la poésie est toujours utilisé aujourd'hui comme exercise de mémorisation (à l'école) et moyen mnémotechnique (dans la publicité).

\section{L'écriture}
L'écriture, véritable épiphanie informatique, permet à un support mort de stoquer un ensemble de données codés sous forme de symboles, donc de repousser d'avantage la frontière de l'espace et du temps. Cette avancé rend assez redondants les techniques de mémorisation lyriques mentionnées ci-dessus. La réaction de ceux qui se sont investies dedans est donc peu surprenant~:
\begin{quote}
Elle ne peut produire dans les âmes, en effet, que l’oubli de ce qu’elles  savent en leur faisant négliger la mémoire. Parce qu’ils auront foi dans  l’écriture, c’est par le dehors, par des empreintes étrangères, et non plus du dedans et du fond d’eux-mêmes, que les hommes chercheront à se ressouvenir. Tu as trouvé le moyen, non point d’enrichir la mémoire, mais de conserver les souvenirs qu’elle a. Tu donnes à tes disciples la présomption qu’ils ont la science, non la science elle-même. Quand ils auront, en effet, beaucoup appris sans maître, ils s’imagineront devenus très savants, et ils ne seront pour la plupart que des ignorants de commerce incommode, des savants imaginaires au lieu de vrais savants.
\end{quote}
Cette critique fait par Platon\cite{plato-phaedrus}, à travers un dialogue entre Socrates et Phaedrus, suggère que l'écriture, en plus de nuire à la mémoire, limite son lecteur à ce que la taxonomie de Bloom\cite{tax-bloom} appelerait \og{}connaître\fg{} à la différence de \og{}comprendre\fg{}, \og{}appliquer\fg{} et cetera.
Ironiquement, si nous connaissons Socrates et Platon c'est grâce à l'écriture. Ce rejet de la vulgarisation de l'information sera répété à travers l'histoire. Cependant rare sont ceux qui, aujourd'hui, pourraient se rappeller de l'Odyssée toute entière~: Platon avait-il donc raison quelquepart? 

\section{Les mathématiques}
Nous pourrions remplir un rapport tout entier s'il était question d'énumérer les apports de l'outil logico-mathématique, donc nous nous limiterons à l'exemple pertinente des tables de logarithmes. Les logarithmes de Napier, introduits en 1614 dans son oeuvre \og{}Mirifici Logarithmorum Canonis Descriptio\fg{}, furent un outil abstrait visant à simplifier des calculs complexes. Les tables de logarithmes sont très vite devenus indispensables pour tout mathématicien, ingénieur, navigateur ou scientifique. D'après Pierre-Simon Laplace cette invention est\cite{history-of-astronomy}~:
\begin{quote}
Un artifice admirable qui, en réduisant à quelques jours un travail de plusieurs mois, double la durée de vie de l'astronome, et lui épargne les erreurs et le dégoût : plaies inséparables des longs calculs.
\end{quote}
Pourtant de nos jours les algorithmes de calcul à base de tables logarithmiques ne sont même pas introduites à l'école~: la popularisation de calculatrices électroniques aux année 1970 leur ont rendus désuets, de même que l'écriture rend inutile l'épopée. D'ailleurs une critique analogue se tient face à la calculatrice. Pour ne donner qu'un exemple, d'après Diane Hunsaker, professeur de mathématiques au Californie\cite{eduworld}~:
\begin{quote}
Les étudiants qui ne font pas la longue-division, qui sortent leur calculatrice pour compléter la réponse, ne comprènent pas le principe soujascente de la division [\ldots] Les calculatrices empêchent les étudiants de voir la structure sousjascente et donc la beauté des mathématiques.
\end{quote}
Si des machines à calculer tels le Boulier existent depuis environ 4000 ans, ceux-ci ne sont véritablement que des supports mémoires~: c'est l'étudiant qui applique l'algorithme permettant de calculer le résultat désiré, ce qui n'est pas le cas pour les machines à calcule mécaniques (tels la Pascaline) ou éléctroniques.


\section{Les technologies du \og{}Broadcast\fg{}~: presse, radio, télévision, internet, \ldots}


%\subsection{Les tournant majeur des technologies de l'information}
//@? Comment introduire la notion de document ?

Les technologies de l'information n'ont eu de cesse de repousser les limites de
la propagation de l'information, 

Avec l'arrivée de l'imprimante, l'Homme est désormais capable de copier
rapidement une information pour la \emph{diffuser}. (Apparition d'une sorte
de broadcast)

Avec le téléphone, l'Homme peut désormais échanger \emph{en tant réelle} une 
information avec une autre personne \og{}~n'importe où~\fg{} dans le monde.

La télévision permet non seulement de faire de même avec des images, mais elle,
ou plutôt son usage, permet bien plus. Elle permet de préformater un document pour ensuite le
diffuser et éventuellement le rediffuser à l'attention d'un public nombreux et
hétérogène tel que l'on peut le faire avec un livre.

Puis vint Internet \og{}~International Network~\fg{}, en vrac: n'importe qui
peut diffuser un document pour un coup très faible, une quantité astronomique
de documents accessibles, Internet se présente comme un méta-outil et offre de
nombreux outils de communication (Mail, Forum, chat, VoIP, conférence, etc...)

\subsection{en vrac}



Depuis les années 90: internet, outil pour tricher?
\url{http://www.apsq.org/sautquantique/doss/d-tricherie.html/}

\ldots

\section{\ldots dans la société}

La notion de nation est-elle pertinente dans un monde connecté par
internet~?

"The Big Switch: Rewiring the World from Edison to Google" 
 \url{http://www.nicholasgcarr.com/bigswitch/}

Nous perdons notre confidentialité de l'intime. Pour Google, Facebook, etc, 
ce n'est pas un problème: ceux qui n'ont rien à cacher n'ont pas à se soucier. Nous
pouvons cependant noter qu'ils ont intérêt à penser ainsi [j'arrive pas à 
trouver l'article qui en parle]. Le voyeurisme de la télé-réalité montre 
que ce phénomène est de Zeitgeist (esprit de l'époque): nous nous soucions 
moins, par exemple, de qui nous entend parler lors de nos
conversations depuis un téléphone portable.

Les réseaux sociaux permettent un exhibitionnisme, et ceci conduit au 
narcissisme:
http://www.guardian.co.uk/technology/2012/mar/17/facebook-dark-side-study-aggressive-narcissism 
[WHY U NO GIVE ME LINK TO ORIGINAL ARTICLE!]

"The Spy in the Coffee Machine: The End of Privacy As We Know It"
 \url{http://eprints.soton.ac.uk/265683/}

En Février 2010 Eben Moglen lance la notion de "Freedom Box" pour faire face à
la centralisation progressive du web. C'est d'ailleurs sa présentation qui fut
l'amorce du projet Diaspora, visant à créer une plateforme sociale décentralisé 
respectant la confidentialité de ses usagers. 

\subsection{\ldots par les individus}


