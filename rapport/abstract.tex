% !TEX encoding = UTF-8 Unicode
% !TEX root = rapport.tex

\chapter*{Abstract}\label{abstract}

\begin{coolquote}[Albert Einstein (attributed)]
\Large Education is what remains when one has forgotten everything one has learnt at school.
\end{coolquote}

\textit{Are memorisation, note-taking and mental calculation really useful abilities that our elders are trying to pass down, or are they simply tools used to sort the intellectual wheat from the chaff? Is education thus a sort of game where one learns skills of little practical use in the "real world"? For even if we consider an individual's intellectual blossoming to be the goal of education, learning facts and figures by heart hardly contributes to this.}

\textit{Perhaps it was once much more pertinent, but today there seems to be an ever-greater chasm opening up between the tools that education provides and what society requires. This is especially true of information-technology, which is shaping our world far faster than education can keep up with. But should it "keep up"? In this age of Information, should education change and, if so, how?}

\textit{In the paper we will be discussing the problem of information-technology and education, as well as various solutions that have been put forward in recent years.}